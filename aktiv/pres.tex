% !TEX TS-program = pdflatex
% !TEX encoding = UTF-8 Unicode

%%% PAGE DIMENSIONS 16:9
\documentclass[notheorems, aspectratio=169]{beamer}

%%% TEXT FEATURES
\usepackage[utf8]{inputenc} % set input encoding (not needed with XeLaTeX)
\usepackage[T2A]{fontenc} % font encoding
\usepackage[russian]{babel} % languages
\usepackage{amsfonts} % for natural, rational, etc numbers
\usepackage{amsmath} % special math symbols
\usepackage{amssymb} % some more special symbols
\usepackage{color, colortbl}
\setbeamertemplate{navigation symbols}{}
\renewcommand{\arraystretch}{1.5}
\usefonttheme{serif}

%%% COLORIZING FEATURES

% color theme settings
\mode<presentation>
{
  \usetheme{AnnArbor}
  \usecolortheme{beaver}
  \setbeamercovered{transparent}
}
\setbeamercolor{frametitle}{use=structure, bg=gray-background, fg=dark}
\setbeamercolor{palette primary}{bg=dark, fg=white}
\setbeamercolor{palette secondary}{bg=light, fg=white}
\setbeamercolor{palette tertiary}{bg=medium, fg=white}

% frame title settings
\setbeamertemplate{frametitle}
{
    \nointerlineskip
    \begin{beamercolorbox}[sep=0.3cm,ht=1.8em,wd=\paperwidth]{frametitle}
        \strut\textbf{\insertframetitle}\strut
        \vskip-1ex%
    \end{beamercolorbox}
}

% colored theorem blocks
\setbeamercolor{block title}{use=structure,fg=white,bg=medium}
\setbeamercolor{block body}{use=structure,fg=black,bg=light!10}

% colored tables
\usepackage{colortbl}
\usepackage{tabularray}

% colored captions
\usepackage{caption}
\DeclareCaptionLabelFormat{mycaption}{\usebeamercolor[medium]{caption name} #1 #2: }
\captionsetup[figure]{labelformat=mycaption, labelsep=none, labelfont=bf}
\captionsetup[table]{labelformat=mycaption, labelsep=none, labelfont=bf}

% colored lists (light-blue squares)
\usepackage{enumitem, xcolor}
\newlist{coloritemize}{itemize}{1}
\setlist[coloritemize]{label=\textcolor{light}{$\blacksquare$}}
\colorlet{itemizecolor}{.}% Default colour for \item in itemizecolor
%%%%%%%%%%%%%%%%%%%%%%%%%%%%%%%%%%%%%%%%%%%%%%%%%%%%%%%%%%%%%%%%%%%%%%%%%%%%%%%
%%% To use a different custom template in the {coloritemize} environment,   %%%
%%% set item parameters:                                                    %%%
%%%                                                                         %%%
%%% \begin{coloritemize}                                                    %%%
%%%     \item[{$\color{medium}\blacksquare$}]                               %%%
%%% \end{coloritemize}                                                      %%%
%%%%%%%%%%%%%%%%%%%%%%%%%%%%%%%%%%%%%%%%%%%%%%%%%%%%%%%%%%%%%%%%%%%%%%%%%%%%%%%

% bordered text
\usepackage[most]{tcolorbox}

\newtcolorbox{mytheorem}[1]{
    enhanced jigsaw,
    colback=red!5!white,
    colframe=red!75!black,fonttitle=\bfseries,
    colbacktitle=red!85!black,enhanced,
    %size=small,%
    %boxrule=1pt,%
    halign title=flush left,%
    %coltitle=blue,%
    breakable,%
    drop fuzzy shadow=black!70!white,%
    left=0pt,
    titlerule=0pt,
    top=1pt,
    bottom=0pt,
    enlarge left by=-0.1cm,
    grow to right by=0.21cm,
    frame empty,
    borderline={0.3mm}{0mm}{dark},
    fonttitle = \bfseries,
    title=#1
}


\newtcolorbox{mypropos}[1]{
    enhanced jigsaw,
    colback=green!5!white,
    colframe=green!75!black,fonttitle=\bfseries,
    colbacktitle=green!65!black,enhanced,
    %size=small,%
    %boxrule=1pt,%
    halign title=flush left,%
    %coltitle=blue,%
    breakable,%
    drop fuzzy shadow=black!70!white,%
    left=0pt,
    titlerule=0pt,
    top=1pt,
    bottom=0pt,
    enlarge left by=-0.1cm,
    grow to right by=0.21cm,
    frame empty,
    borderline={0.3mm}{0mm}{dark},
    fonttitle = \bfseries,
    title=#1
}


\newtcolorbox{myexample}[1]{
    enhanced jigsaw,
    colback=blue!5!white,
    colframe=blue!75!black,fonttitle=\bfseries,
    colbacktitle=blue!65!black,enhanced,
    %size=small,%
    %boxrule=1pt,%
    halign title=flush left,%
    %coltitle=blue,%
    breakable,%
    drop fuzzy shadow=black!70!white,%
    left=0pt,
    titlerule=0pt,
    top=1pt,
    bottom=0pt,
    enlarge left by=-0.1cm,
    grow to right by=0.21cm,
    frame empty,
    borderline={0.3mm}{0mm}{dark},
    fonttitle = \bfseries,
    title=#1
}

%%% CUSTOM COLORS
% presentation pack
\definecolor{light}{RGB}{34, 187, 221}
\definecolor{medium}{RGB}{0, 85, 170}
\definecolor{dark}{RGB}{10, 32, 64}
\definecolor{gray-background}{RGB}{237, 237, 237}
\definecolor{dark-gray}{RGB}{167, 176, 184}

% plots pack 
\definecolor{myorange}{RGB}{221, 69, 34}
\definecolor{mymagenta}{RGB}{162, 34, 221}
\definecolor{mygreen}{RGB}{33, 221, 34}

%%% IMAGES AND PLOTS
\usepackage{tikz}
\usepackage[]{graphicx} % Required for including images
\usepackage{pgfplots}
\pgfplotsset{compat=1.18}
\pgfplotsset{every axis legend/.append style={anchor=north east, font=\footnotesize}} %legend style

%%% DECORATIVE ELEMENTS
\usepackage{pdfpages} % add title as pdf file

% date
\setbeamertemplate{page number in head/foot}[totalframenumber]
\date{июнь 2025 г.}

% logo
\logo{
\begin{tikzpicture}[overlay,remember picture]
\node[left=0cm] at (current page.-26){
    \includegraphics[width=1.2cm]{AKTIV_NEW LOGO.png}
};
\end{tikzpicture}
}

%%% READ MORE
%%% https://www.cpt.univ-mrs.fr/~masson/latex/Beamer-appearance-cheat-sheet.pdf
%%% ___________________________________________________________________________

\newtheorem{definition}{Определение}
\newtheorem{theorem}{Теорема}
\newtheorem{lemma}{Лемма}
\newtheorem{corollary}{Следствие}
\newtheorem{prop}{Утверждение}
\newtheorem{example}{Пример}

\newtheoremstyle{named}{}{}{\itshape}{}{\bfseries}{.}{.5em}{\thmnote{#3's }#1}
\theoremstyle{named}
\newtheorem*{namedtheorem}{Теорема}

\graphicspath{{./images/}}
\setbeamerfont{footnote}{size=\tiny}

%\usepackage[c]{beamerfontthemestructureitalicserif}

%%%%%%% FONTS %%%%%%%%%%%%%%%%

% \setsansfont{FiraSans-Light}[
% Path = ./styles/fonts/,
% Extension = .otf,
% BoldFont=FiraSans-Regular,
% ItalicFont=FiraSans-LightItalic,
% BoldItalicFont=FiraSans-Italic
% ]
% % Courier New
% \setmonofont{FiraMono-Medium}[
% Path = ./styles/fonts/,
% Extension = .otf,
% BoldFont=FiraMono-Bold,
% ItalicFont=FiraMono-Medium,
% BoldItalicFont=FiraMono-Bold,
% ]

% \newfontfamily{\cyrillicfont}{FiraMono-Medium}[
% Path = ./styles/fonts/,
% Extension = .otf,
% BoldFont=FiraMono-Bold,
% ItalicFont=FiraMono-Medium,
% BoldItalicFont=FiraMono-Bold,
% ]

% \usepackage{unicode-math}
% \setmathfont{STIX2Math}[
% Path = ./styles/fonts/,
% Extension = .otf
% ]


%\usepackage[style=verbose,backend=biber]{biblatex}
\usepackage[style=authortitle-comp,backend=bibtex]{biblatex}
\addbibresource{matis_biblio.bib}

\usepackage{csquotes}

% ------------- Creating a new block template ----------
% theorem block 
\makeatletter
\def\th@newblock{%
  \normalfont 
  \def\inserttheoremblockenv{theoremblock}}
\theoremstyle{newblock}
\newtheorem{Thm}[theorem]{Теорема}
\makeatother

% lemma block 
\makeatletter
\def\th@newblock{%
  \normalfont 
  \def\inserttheoremblockenv{theoremblock}}
\theoremstyle{newblock}
\newtheorem{Lem}[theorem]{Лемма}
\makeatother

% example block 
\makeatletter
\def\th@newblock{%
  \normalfont 
  \def\inserttheoremblockenv{theoremblock}}
\theoremstyle{newblock}
\newtheorem{Def}[theorem]{Определение}
\makeatother

% remark block 
\makeatletter
\def\th@newblock{%
  \normalfont 
  \def\inserttheoremblockenv{theoremblock}}
\theoremstyle{newblock}
\newtheorem{Rem}[theorem]{Замечание}
\makeatother




%%%%%%%%%%%%% ADDITIONAL COMMANDS %%%%%%%%%%%%%%%

% множества
\newcommand{\EE}{\mathbb{E}}
\newcommand{\BB}{\mathbf{B}}
\newcommand{\NN}{\mathbb{N}}
\newcommand{\ZZ}{\mathbb{Z}}
\newcommand{\FF}{\mathbb{F}}
% группы, квазигруппы
\newcommand{\SSS}{\mathcal{S}}
\newcommand{\sprop}{\mathcal{S}^{\mathsf{prop}}}
\newcommand{\QQQ}{Q_1 \times \ldots \times Q_n}
\newcommand{\GGG}{G}

% функции и аргументы
\newcommand{\ff}{\mathcal{F}}
\newcommand{\gf}{\mathcal{G}}
\newcommand{\hf}{\mathcal{H}}
\renewcommand{\ss}{\mathbf{s}}
\newcommand{\xx}{\mathbf{x}}
\newcommand{\yy}{\mathbf{y}}
%\newcommand{\zz}{\mathbf{z}}
\newcommand{\uu}{\mathbf{u}}
\newcommand{\vv}{\mathbf{v}}
\newcommand{\ww}{\mathbf{w}}
\newcommand{\dd}{\partial}
\newcommand{\loc}{\mathsf{loc}}
\newcommand{\rec}{\mathsf{rec}}
\newcommand{\hupf}{\mathsf{HUFP}}
\newcommand{\divides}{\mid}

% команды 
\newcommand{\proj}{\Pi}
\newcommand{\inv}{\mathsf{inv}}
\newcommand{\Img}{\mathsf{Im}}
\newcommand{\fib}{\mathsf{Fib}}
\newcommand{\lucas}{\mathsf{Lucas}}
\newcommand{\uso}{\mathsf{USO}}


% криптография
\newcommand{\sample}{\gets^{\mathcal{U}}}
\newcommand{\Dom}{\mathbf{Dom}}
\newcommand{\Keys}{\mathbf{Keys}}
\newcommand{\Twk}{\mathbf{Twk}}
\newcommand{\enc}{\mathsf{Enc}}
\newcommand{\dec}{\mathsf{Dec}}
\newcommand{\gen}{\mathsf{Gen}}
\newcommand{\fpe}{\mathsf{FPE}}

%%%%%%%%%%%%%%%%%%%%%%%%%%%%%%%%%%%%%%%%%%%%%%%%%


\AtBeginSection[]
{
    \begin{frame}
        \frametitle{Содержание}
        \tableofcontents[currentsection]
    \end{frame}
}

%%%%%%%%%%%%%%%%%%%%%%%%%%%%%%%

%!TEX root = ./pres.tex

\title[Правильные семейства]
{Правильные семейства функций и порождаемые ими
квазигруппы}
\subtitle{Комбинаторные и алгебраические свойства}
\date{июнь 2025 г.}
\author{К.~Царегородцев\inst{1, 2}}

\institute[МГУ, Актив]
{
  \inst{1}%
  МГУ им. М.~В.~Ломоносова
  \and
  \inst{2}%
  АО <<Актив-софт>>\\
  Москва, Россия
}


\begin{document}

%\maketitle

%%% TITLE
\includepdf[pages=1]{title.pdf}

\begin{frame}{Содержание доклада}
    \setbeamertemplate{section in toc}[sections numbered]
    \tableofcontents%[hideallsubsections]
\end{frame}

%!TEX root = ./pres.tex

\section{Введение: квазигруппы в криптографии}


\begin{frame}{Технический момент: используемые обозначения}
    \begin{table}
        \begin{center}
            \begin{tabular}{|>{\columncolor{Gray}}c|c|}
                \hline
                $Q$ &квазигруппа с операцией $\circ$ \\
                \hline 
                $k$ & размер множества $Q$, $k = \lvert Q \rvert$, значность логики \\
                \hline
                $\EE_k$ & множество $\{0, \ldots, k-1 \}$ (обычно предполагаем $\EE_k = \ZZ_k$) \\
                \hline
                $\ff$ & семейство (набор) функций $\ff = (f_1, \ldots, f_n)$, \\
                      & $\ff \colon Q^n \to Q^n$ \\
                \hline 
                $f_i$ & $i$-я функция семейства $\ff$ \\
                \hline
                $n$ & размер семейства \\
                \hline
                $Func(Q)$ & множество функций $f \colon Q \to Q$ \\
                \hline 
                $Perm(Q)$ & множество подстановок (биекций) на $Q$ \\
                \hline
            \end{tabular}
        \end{center}
    \end{table}
\end{frame}


\begin{frame}{ \textquote{Обычная} криптография}
    \begin{coloritemize}
        \item Часто используемые алгебраические структуры в криптографии: поля, кольца (коммутативные, ассоциативные, с единицей),  коммутативные группы, коды, решетки.
        \pause 
        \item В исследовательской литературе предлагаются к рассмотрению и более \textquote{экзотические} структуры: \textbf{некоммутативные} группы и алгебры (например, группы кос, алгебры матриц, алгебра кватернионов и так далее)~\footcite{myasnikov2011non, romankov, moldovyan}, \textbf{неассоциативные структуры}: квазигруппы, квазигрупповые кольца и т.д~\footcite{glukhov, artamonov18, markov2020nonassociative}.
        \pause 
        \item В докладе будет рассматриваться один способ построения квазигрупп на основе т.н. \textquote{правильных семейств функций}, а также свойства этих семейств самих по себе.
    \end{coloritemize}
\end{frame}


\begin{frame}
    \begin{myexample}{Квазигруппа}
        Множество $Q$ с заданной на нём бинарной операцией
        \(
          \circ \colon Q \times Q \to Q, 
        \)
        со следующим свойством: для любых $a, b \in Q$ существуют единственные $x, y \in Q$, такие что:
        \[
          a \circ x = b, \qquad y \circ a = b.
        \]
        \footcitetext{belousov, keedwell}
    \end{myexample}

    \pause
    Другими словами, операции \textbf{левого} $L_a$ и \textbf{правого} $R_a$ умножения (сдвиги)
    \begin{gather*}
        L_a \colon Q \to Q,\, L_a(x) = a \circ x, \; R_a \colon Q \to Q,\, R_a(y) = y \circ a,
    \end{gather*}
    являются биекциями на $Q$.
    
    \pause 
    По сути = группа без ассоциативности и единицы, но \textbf{с сокращением} как слева, так и справа.
\end{frame}


\begin{frame}{Несколько примеров}
    \begin{coloritemize}
        \item $Q$~--- любая группа, например $Q = \ZZ_k$, $\circ = +$; 
        \item $Q = \ZZ_k$, $\circ = -$ (не группа, т.к. $a - (b - c) \ne (a - b) - c$);
        \item $(G, \cdot)$~--- группа, $\pi$, $\sigma$, $\tau$~--- подстановки на $G$, тогда можно рассмотреть \textbf{изотоп~\footcite{belousov, keedwell}}:
        \[
            x \circ y = \tau(\pi(x) \cdot \sigma(y)).
        \]
    \end{coloritemize}
\end{frame}


\begin{frame}{Латинский квадрат}
    \begin{coloritemize}
        \item Квадратная таблица размера $k \times k$, заполнена элементами множества $\{ 0, \ldots, k-1 \}$, каждое элемент появляется \textbf{только один раз} в каждом столбце и каждой строке таблицы.
        \item Таблица умножения квазигруппы $Q = \{q_1, \ldots, q_k \}$ (на пересечении $i$-й строки и $j$-го столбца пишем $(q_i \circ q_j) \in Q$) является латинским квадратом.
    \end{coloritemize}
    \begin{columns}[T] % gather columns
        \begin{column}{.2\textwidth}
            \(
                \begin{bmatrix}
                    0 & 1 & 2 & 3 & 4 \\
                    1 & 0 & 3 & 4 & 2 \\
                    2 & 3 & 4 & 0 & 1 \\
                    3 & 4 & 1 & 2 & 0 \\
                    4 & 2 & 0 & 1 & 3 \\
                \end{bmatrix}
            \)
        \end{column}%
        \hfill%
        \begin{column}{.4\textwidth}
            \begin{figure}[h]
                \centering 
                \includegraphics[scale = 0.18]{fisher.jpg}
            \end{figure}
        \end{column}%
    \end{columns}  
\end{frame}


\begin{frame}{Квазигруппы: симметричные механизмы}
    \begin{coloritemize}
        \item ГПСЧ~\footcite{dimitrova2004quasigroup, markovski2005unbiased}, блочные шифры~\footcite{inru}, поточные шифры~\footcite{edon80}, хэш-функции~\footcite{gligoroski2008edon, gligoroski2009family, EdonR, EdonRprime, mileva2009quasigroup}.
        \pause 
        \item Низкоресурсные (lightweight) хэш-функция и AEAD-алгоритм~\footcite{otte2019gage, gligoroski2019s}.
        \pause 
        \item Основным нелинейным компонентом в упомянутых алгоритмах является квазигрупповое умножение.
        \pause 
        \item Приложения в теории кодирования~\footcite{nechaev98, nechaev04, couselo2004loop, markov12, markov2020nonassociative}.
    \end{coloritemize}
\end{frame}


\begin{frame}{Квазигруппы: асимметричные механизмы}
    \begin{coloritemize}
        \item Асимметричные схемы подписи~\footcite{gligoroski2008public, gligoroski2008multivariate, chen2010multivariate, gligoroski2011mqq}~--- аналоги пост-квантовых схем многомерной криптографии (multivariate cryptography).
        \pause
        \item Схемы~--- аналоги протокола Диффи-Хеллмана выработки общего ключа~\footcite{katyshev14, katyshev18}, гомоморфное шифрование~\footcite{gribov2010construction, gribov15, markov20}: используются \textbf{ППС/ПЛС-группоиды}, \textbf{луповые кольца} над медиальными квазигруппами (изотопы абелевых групп с коммутирующими автоморфизмами).
        \pause 
        \item Более подробно вопрос освещен в~\footcite{glukhov, artamonov18, shcherbacov2017elements, chauhan2021quasigroups}.
    \end{coloritemize}
\end{frame}


\begin{frame}{Как задать квазигруппу?}
    \begin{coloritemize}
        \item В общем случае квазигруппа над множеством $Q$ задается таблицей умножения размера $\lvert Q \rvert \times \lvert Q \rvert$; это много.
        \pause 
        \item Случайная генерация (поиск + отсев) квазигрупп из некоторого узкого класса с компактно задаваемыми представителями~\footcite{gligoroski2008public, chen2010multivariate}.
        \pause 
        \item Итеративное построение из более \textquote{маленьких} (конструкции наподобие прямых произведений)~\footcite{gribovphd, EdonRprime}.
        \pause 
        \item Изотопы некоторых \textquote{хорошо изученных} групп (например, изотоп группы точек эллиптической кривой~\footcite{DH16}, модульное вычитание~\footcite{snavsel2009hash}).
        \pause 
        \item Функциональное задание квазигруппы: поговорим о нём подробнее.
    \end{coloritemize}
\end{frame}


\begin{frame}{Функциональное задание квазигруппы}
    \begin{coloritemize}
        \item Можно перейти от табличного задания операции к функциональному~\footcite{nosov08}: 
        \[
            x \circ y = z \leftrightarrow z_i = f_i(x_1, \ldots, x_n, y_1, \ldots, y_n). 
        \]
        \pause 
        \item Для краткости набор функций $\ff = (f_1, \ldots, f_n)$, $f_i \colon Q^n \to Q, \quad i = 1, \ldots, n$, будем называть семейством функций; семейство задает отображение множества $Q^n$ в себя.
        \pause
        \item Рассмотрим для простоты случай $Q_i = \{0, 1\}$: какие условия надо наложить на функции $f_i$, чтобы операция $x \circ y$ задавала \textbf{структуру квазигруппы} на $\{0, 1\}^n$?
    \end{coloritemize}
\end{frame}









\section{Правильные семейства функций и квазигруппы}










\begin{frame}{Правильные семейства булевых функций}
    \begin{myexample}{Правильное семейство}
        Семейство булевых функций $f_i \colon \EE_2^n \to \EE_2^n$ называется правильным, если для любых двух наборов $x \ne y$ найдется такая координата $i$, что $x_i \ne y_i$, но $f_i(x) = f_i(y)$.
        \footcitetext{nosov98, nosov99}
    \end{myexample}
    \pause 

    Правильные семейства можно задавать над логикой любой значности $k$~\footcite{nosov06}, над произвольными группами~\footcite{nosov06abel}; над прямыми произведениями других квазигрупп~\footcite{galatenko2020latin} и $d$-квазигрупп~\footcite{plaksina14}.
\end{frame}


\begin{frame}{Примеры правильных семейств}
    \begin{coloritemize}
        \item Константные семейства $f_i \equiv const_i$ являются правильными.
        \pause 
        \item Треугольные семейства являются правильными
        \[
            \begin{bmatrix}
                f_1 \\
                f_2 \\
                % f_3 \\
                \vdots \\
                f_n 
            \end{bmatrix}
            =
            \begin{bmatrix}
                f_{1}( \, ) \\
                f_{2}(x_{1}) \\
                % f_{3}(x_{1}, x_{2}) \\
                \vdots \\
                f_{n}(x_{1}, \ldots, x_{n-1})
            \end{bmatrix}.
        \]
        \pause 
        \item Из определения правильности следует, что $f_i$ не зависит существенно от $x_i$.
        \footcitetext{nosov98, nosov06abel}
    \end{coloritemize}
\end{frame}


\begin{frame}
    \begin{mypropos}{Класс квадратичных семейств}
        Семейство $\ff$ является правильным для любого $n \ge 1$:
        \[
            \ff(x_1, \ldots, x_n) = 
            \begin{bmatrix}
                0 \\
                x_1 \\
                x_1 \oplus x_2 \\
                \vdots \\
                x_1 \oplus x_2 \oplus \ldots \oplus x_{n-1}
                \end{bmatrix}
                \bigoplus
                \begin{bmatrix}
                \bigoplus_{i < j, \; i, j \ne 1}^n \; x_i x_j \\
                \bigoplus_{i < j, \; i, j \ne 2}^n \; x_i x_j \\
                \bigoplus_{i < j, \; i, j \ne 3}^n \; x_i x_j \\
                \vdots \\
                \bigoplus_{i < j, \; i, j \ne n}^n \; x_i x_j \\
            \end{bmatrix}.
        \]
    \end{mypropos}
\end{frame}


\begin{frame}{Число правильных булевых семейств $T(n)$}
    \begin{center}
        \begin{tabular}{|c|c|}
            \hline
            \rowcolor{Gray}
            Размер $n$ & $T(n)$ \\
            \hline
            $n = 1$ & 2 \\
            \hline
            $n = 2$ & 12 \\
            \hline
            $n = 3$ & 744\\
            \hline
            $n = 4$ & 5541744 \\
            \hline
            $n = 5$ & 638560878292512 \\
            \hline
        \end{tabular}
    \end{center}

    \begin{mypropos}{Оценка на число булевых правильных семейств}
        \[
            n^{A \cdot 2^n} \le T(n) \le n^{B \cdot 2^n},
        \]
        где $A$, $B$~--- некоторые положительные константы.
        \footcitetext{numberUSO}
    \end{mypropos}
\end{frame}


\begin{frame}{Сложность распознавания правильности}
    \begin{coloritemize}
        \item В общем случае проверка правильности является сложной задачей: если семейство задано в форме КНФ, то задача проверки правильности coNP-полна~\footcite{nosov98}.
        \pause 
        \item В определенных случаях задача проверки правильности может быть упрощена, в частности, за счет вида графа существенной зависимости~\footcite{rykov14}.
        \pause 
        \item Алгоритм проверки правильности булева семейства требует порядка $\Theta(4^n)$ операций вычисления правильного семейства на двоичном наборе $x$ (проверка по определению правильности).
    \end{coloritemize}
\end{frame}


\begin{frame}{Один способ задания квазигруппы}
    \begin{coloritemize}
        \item Есть несколько способов задать структуру квазигруппы на множестве $Q^n$ с помощью правильных семейств, в докладе рассмотрим одно из них.
        \pause 
        \item Пусть $\ff$, $\gf$~--- два правильных семейства функций размера $n$ над группой $(\GGG^n, +)$.
        Для $\xx, \yy \in \GGG^n$ зададим операцию $\circ$ следующим образом:
        \[
            \xx \circ \yy = \xx + \ff(\xx) + \yy + \gf(\yy).
        \]
        \pause 
        \item Операция $\circ$ является квазигрупповой.
        \pause 
        \item Это следует из более общей теоремы об эквивалентности свойства правильности семейства и регулярности некоторого семейства отображений.
    \end{coloritemize}
\end{frame}


\begin{frame}{Криптографические свойства квазигрупп}
    \begin{coloritemize}
        \item Малое число ассоциативных троек, то есть троек элементов $(a, b, c) \in Q^3$
        \[
            (a \circ b) \circ c = a \circ (b \circ c).
        \]
        Количество таких троек называется индексом ассоциативности.
        \pause 
        \item Отсутствие подквазигрупп, т.е. подмножеств $Q' \subset Q$, которые замкнуты относительно умножения.
        \pause 
        \item Полиномиальная полнота квазигрупп (любое отображение $f \colon Q^n \to Q$ задается с помощью композиции констант и операции умножения).
    \end{coloritemize}
\end{frame}


\begin{frame}{Индекс ассоциативности, теория}
    \[
        \xx \circ \yy = \xx + \ff(\xx) + \yy + \gf(\yy).
    \]
    \begin{mypropos}{Об индексах ассоциативности}
        \begin{coloritemize}
            \item Индексы ассоциативности квазигрупп, построенных по паре $(\ff, \gf)$ и по паре $(\gf, \ff)$, совпадают.
            \item Для $G = \ZZ_2$ индексы ассоциативности квазигрупп, построенных по паре $(\ff, \gf)$ и по паре $(\ff \oplus \alpha, \gf \oplus \alpha)$, совпадают.
            \item Для $G = \ZZ_2$ количество ассоциативных троек в квазигруппе, построенной по паре правильных булевых семейств $(\ff, \gf)$, четно.
        \end{coloritemize}
    \end{mypropos}
\end{frame}



\begin{frame}{Индекс ассоциативности, эксперимент $n=2$}
    \[
        (x, y) \to x \circ y = x + \ff(x) + y + \gf(y).
    \]
    \begin{center}
        \begin{tabular}{|c|c|}
            \hline
            \rowcolor{Gray}
            $a(Q)$ & Кол-во $Q$ \\
            \hline
            16 & 32 \\
            \hline
            32 & 96 \\
            \hline
            64 & 16 \\
            \hline
        \end{tabular}
    \end{center}
\end{frame}


\begin{frame}{Индекс ассоциативности, эксперимент $n=3$}
    \begin{center}
        \begin{tabular}{|c|c|c|c|}
            \hline
            \rowcolor{Gray}
            $a(Q)$ & Кол-во $Q$ & $a(Q)$ & Кол-во $Q$ \\
            \hline
            64 & 27648 & 144 & 3072\\
            80 & 103424 & 160 & 84480 \\
            88 & 18432 & 176 & 6144\\
            96 & 82944 & 192 & 18432\\
            104 & 33792 & 208 & 3072\\
            112 & 21504 & 256 & 10368\\
            120 & 21504 & 320 & 2304\\
            128 & 116352 & 512 & 64\\
            \hline
        \end{tabular}
    \end{center}
\end{frame}


\begin{frame}{Индекс ассоциативности, эксперимент $n=4$}
    \begin{columns}[T] % gather columns
        \begin{column}{.49\textwidth}
            \begin{figure}[h]
                \centering 
                \includegraphics[scale = 0.4]{histogram.png}
            \end{figure}
        \end{column}%
        \hfill%
        \begin{column}{.49\textwidth}
            \begin{figure}[h]
                \centering 
                \includegraphics[scale = 0.4]{density.png}
            \end{figure}
        \end{column}%
    \end{columns}  
\end{frame}


\begin{frame}{Полиномиальная полнота, теория}
    \begin{coloritemize}
        \item Пусть $\ff \colon Q^n \to Q^n$~--- правильное, $(Q, \circ)$~--- квазигруппа.
        Введем обозначение $\sigma_{\ff} \in Perm(Q^n)$:
        \[ 
            \sigma_\ff(x) \colon x \to x \circ \ff(x),
            \quad
            \begin{bmatrix}
                x_1 \\
                \vdots \\
                x_n
            \end{bmatrix} 
            \to 
            \begin{bmatrix}
                x_1 \circ f_1(x_1, \ldots, x_n) \\
                \vdots \\
                x_n \circ f_n(x_1, \ldots, x_n)
            \end{bmatrix}
        \]
        \pause 
        \item Для изучения полиномиальной полноты нужно, в частности, хорошо понимать свойства подстановок $\sigma_{\ff}$.
    \end{coloritemize}
\end{frame}


\begin{frame}{Полиномиальная полнота, теория}
    \[
        \sigma_\ff(x) \colon x \to x \circ \ff(x),
    \]
    \begin{coloritemize}
        \item Если $(Q, \circ)$~--- группа (т.е., операция $\circ$ ассоциативна), то множество \textquote{правильных подстановок} замкнуто относительно взятия обратного элемента (в случае, когда $Q$~--- группа).
        \pause 
        \item Множество \textquote{правильных подстановок} $\sprop$ \textbf{не является} подгруппой $Perm(Q^n)$.
        \pause 
        \item Замыкание $\sprop$ действует транзитивно на $Q^n$.
        \pause 
        \item При $Q = \EE_2$ замыкание $\sprop$ порождает все множество подстановок $Perm(\EE_2^n)$.
        \pause 
        \item У подстановки $\sigma_{\ff}(x) = x \oplus {\ff}(x)$ чётное число неподвижных точек.
    \end{coloritemize}
\end{frame}



\begin{frame}{Полиномиальная полнота, эксперимент $n=2$}
    \[
        \xx, \yy \in \ZZ_2^n \quad \xx \circ \yy = \xx \oplus \ff(\xx) \oplus \yy \oplus \gf(\yy).
    \]

    \begin{center}
        \begin{tabular}{|>{\columncolor{Gray}}c|c|c|}
            \hline
            \rowcolor{Gray}
            Свойства & Афинная & Неаффинная \\
            \hline
            Не простая & 112 & 0 \\
            \hline
            Простая & 32 & \textbf{0} \\
            \hline
        \end{tabular}
    \end{center}
\end{frame}


\begin{frame}{Полиномиальная полнота, эксперимент $n=3$}
    \[
        \xx, \yy \in \ZZ_2^n \quad \xx \circ \yy = \xx \oplus \ff(\xx) \oplus \yy \oplus \gf(\yy).
    \]

    \begin{center}
        \begin{tabular}{|>{\columncolor{Gray}}c|c|c|}
            \hline
            \rowcolor{Gray}
            Свойства & Афинная & Неаффинная \\
            \hline
            Не простая & 30784 & 231936 \\
            \hline
            Простая & 9216 & \textbf{281600} \\
            \hline
        \end{tabular}
    \end{center}
\end{frame}









\section{Свойства правильных семейств функций}








\begin{frame}
    \begin{mypropos}{Преобразование сдвига}
        \label{thm:shift}
        Для любого $\alpha = (a_1, \ldots, a_n) \in Q^n$ определим преобразование сдвига:
        \begin{gather*}
            x \in Q^n \to L_{\alpha}(x) = (a_1 \circ x_1, \ldots, a_n \circ x_n), \\
            x \in Q^n \to R_{\alpha}(x) = (x_1 \circ a_1, \ldots, x_n \circ a_n).
        \end{gather*}
        Если $\ff \colon Q^n \to Q^n$ правильное, то $T_{\alpha}(\ff(T_{\beta}(x)))$ также правильное, где $T \in \{L, R\}$, $\alpha, \beta \in Q^n$.
    \end{mypropos}
    Обобщение результата~\footcite{nosov06abel} для абелевых групп.
\end{frame}


\begin{frame}
    \begin{mypropos}{Преобразование перекодировки}
        \label{thm:reencoding}
        Для любого набора $\Psi = (\psi_1, \ldots, \psi_n) \in Func(Q)^n$ определим преобразование перекодировки:
        \[
            x \in Q^n \to \Psi(x) = (\psi_1(x_1), \ldots, \psi_n(x_n)).
        \]
        Пусть $\Phi \in Func(Q)^n$, $\Psi \in Perm(Q)^n$.
        Если $\ff(x) = (f_1(x), \ldots, f_n(x))$ правильное, то $\Phi(\ff(\Psi(x)))$ также правильное.
        \footcitetext{galatenko21criterion}
    \end{mypropos}
    \pause
    Если $\Phi, \Psi \in Perm(Q)^n$, то подобные преобразования будем называть преобразованиями перекодировки.
    \pause
    
    Сдвиги являются частными случаями преобразования перекодировки.
\end{frame}


\begin{frame}
    \begin{mypropos}{Согласованная перенумерация}
        Пусть $\sigma \in Perm(n)$, определим преобразование согласованной перенумерации:
        \begin{gather*}
            \ff \to \sigma(\ff), \\
            f_i(x_1, \ldots, x_n) \to 
            f_{\sigma(i)}(x_{\sigma(1)}, \ldots, x_{\sigma(n)}).
        \end{gather*}
        Если $\ff(x)$~--- правильное, то $\sigma(\ff)$ также правильное.
        \footcitetext{nosov06abel}
    \end{mypropos}
\end{frame}


\begin{frame}
    \begin{mypropos}{Проекция}
        Подставим значение $a \in Q$ вместо переменной $x_i$ и исключим функцию $f_i$, $1 \le i \le n$.
        \[
            F'(x_1,\ldots,x_{i-1}, x_{i+1}, \ldots, x_n) = \proj^i_a(F) =
            \begin{bmatrix}
              f_1(x_1,\ldots,x_{i-1}, a, x_{i+1}, \ldots, x_n) \\
              \vdots \\
              f_{i-1}(x_1,\ldots,x_{i-1}, a, x_{i+1}, \ldots, x_n) \\
              f_{i+1}(x_1,\ldots,x_{i-1}, a, x_{i+1}, \ldots, x_n) \\
              \vdots \\
              f_{n}(x_1,\ldots,x_{i-1}, a, x_{i+1}, \ldots, x_n) \\
            \end{bmatrix}.
        \]
        Полученное семейство является правильным.
        \footcitetext{galatenko2020latin}
    \end{mypropos}
\end{frame}


\begin{frame}{Общий вид биекций, сохраняющих правильность}
    Сдвиги, согласованные перенумерации, перекодировки~--- все эти преобразования:
    \begin{coloritemize}
        \item биективны,
        \pause 
        \item сохраняют правильность семейства,
        \pause 
        \item являются изометриями $\EE_k^n$ (в метрике Хэмминга).
    \end{coloritemize}

    \pause
    Общая постановка задачи: пусть $\Phi$, $\Psi$~--- биекции на $Q^n$: $\Phi, \Psi \in Perm(Q^n)$.
    Рассмотрим стабилизатор множества всех правильных семейств, заданных на $Q^n$:
    \[
        \{(\Phi, \Psi) \in Perm(Q^n) \mid \Phi(F(\Psi(x))) \text{ правильно для любого правильного } F \colon Q^n \to Q^n \}.  
    \]
        
    \pause
    Описать структуру этого множества.
\end{frame}


\begin{frame}{Общий вид биекций, сохраняющих правильность}
    \begin{mypropos}{Стабилизатор правильных семейств}
        Пусть семейства $\gf(\xx)$ вида $\gf(\xx) = \Phi(\ff(\Psi(\xx)))$ являются правильным для всех правильных семейств $\ff$, заданных на $\EE_k^n$, $\Phi$ и $\Psi$~--- биекции множества $\EE_k^n$.
        Тогда $\Phi$ и $\Psi$ имеют вид 
        \[
            \Phi = \sigma \circ A, \Psi = \sigma \circ B, 
        \]
        где использованы следующие обозначения:
        \begin{description}
            \item[$\sigma \in \SSS_n$:] перенумерация координат вектора,
            \item[$A, B \in \left( \SSS_{\EE_k} \right)^n$:] перекодировки вектора. 
        \end{description}
    \end{mypropos}
\end{frame}


\begin{frame}{Неподвижные точки правильного семейства}
    \begin{mypropos}{Неподвижные точки}
        Булево семейство $\ff$ является правильным тогда и только тогда, когда семейство $\ff$ и каждая из его проекций имеет единственную неподвижную точку.
    \end{mypropos}
    \pause 

    Это свойство задает соответствие между правильными булевыми семействами и двумя другими объектами: $\uso$-ориентациями булевых кубов (используются в задачах оптимизации~\footcite{USOphd}) и $\hupf$-сетями (используются в задачах математической биологии~\footcite{thomas1991regulatory, richard2015fixed, ruet2015asynchronous, ruet2016local}).
\end{frame}


\begin{frame}{Неподвижные точки правильного семейства}
    \begin{coloritemize}
        \item Полученные соответствия позволяют перевести (с обобщением) часть результатов, полученных в контексте оптимизации или мат. биологии на язык правильных семейств: например, вероятностный алгоритм порождения правильных семейств с помощью процедуры MCMC~\footcite{galatenko21generation}, оценка на число булевых правильных семейств~\footcite{dm21}, новые классы правильных семейств.
        \pause
        \item В общем случае более общий критерий: семейство $\ff \colon \EE_k^n \to \EE_k^n$ является правильным тогда и только тогда, когда для любой перекодировки $\ff$ все её проекции имеют единственную неподвижную точку~\footcite{galatenko21criterion}.
    \end{coloritemize}
\end{frame}


\begin{frame}{Характеризация через несамодвойственные проекции}
    
    Отображение $\ff \colon \EE_2^n \to \EE_2^k$ самодвойственно, если для любого набора $x \in \EE_2^n$ выполняется свойство $\ff(\overline{x}) = \overline{\ff(x)}$.

    \pause 
    \begin{mypropos}{О несамодвойственности проекций} 
        Семейство $\ff$ булевых функций правильно тогда и только тогда, когда каждая из его проекций 
        \[
            \proj^{a_1, \ldots, a_k}_{i_1, \ldots, i_k}(\ff)
        \] 
        \textbf{не является} самодвойственным булевым отображением.
    \end{mypropos}
    
    Этот результат позволяет снизить сложность проверки правильности с $\Theta(4^n)$ операций вычисления семейства в точке до $\Theta(3^n)$ операций.
\end{frame}


\begin{frame}{Кликовое представление правильных семейств}
    \begin{coloritemize}
        \item Правильные семейства находятся во взаимно-однозначном соответствии с кликами некоторым образом построенного графа (\textquote{обобщенный граф Келлера}).
        \pause
        \item Для $k=2$ перенос из теории $\mathsf{USO}$-ориентаций~\footcite{borzechowski2022universal}, для $k > 2$~--- авторское обобщение.
        \pause
        \item Обобщенный граф Келлера $G(k, n)$: $V = \EE_{k^2}^n$, 
        \[
            \{v, w\} \in E \leftrightarrow \exists i, \, 1 \le i \le n \colon v_i \equiv w_i \text{ mod } \; k, \; v_i \ne w_i.
        \]
        \pause
        \item Графы примечательны тем, что в случае $k = 2$ некоторым образом кодируют неэквивалентные замощения пространства гиперкубами~\footcite{sikiric2007cube, mathew2013enumerating}.
    \end{coloritemize}
    \pause
    \begin{mypropos}{Кликовое представление правильных семейств}
        Каждой клике на $k^n$ вершинах в графе $G(k, n)$ можно поставить в биективное соответствие некоторое правильное семейство $\ff_n$ размера $n$ на $\EE_k^n$.
    \end{mypropos}
\end{frame}


%%% LAST SLIDE
\includepdf[pages=1]{final.pdf}

%%% BIBLIOGRAPHY
\begin{frame}[allowframebreaks]{Список литературы}
    \printbibliography
\end{frame}

\end{document}

