%!TEX root=../nosov.tex

\begin{frame}
    \begin{myexample}{Правильное семейство}
        Семейство функций 
        \[
            f_i \colon Q^n \to Q, \quad 1 \le i \le n,
        \]
        называется правильным, если для любых двух наборов $x \ne y$ найдется индекс~$i$, что $x_i \ne y_i$, но $f_i(x) = f_i(y)$.
        \footcitetext{nosov98, nosov99,nosov06abel,galatenko2020latin}
    \end{myexample}
\end{frame}


\begin{frame}{Некоторые примеры правильных семейств}
    \begin{coloritemize}
        \item Константные семейства: $f_i \equiv const_i$
        \pause 
        \item Треугольные семейства:
        \[
            \begin{bmatrix}
                f_1 \\
                f_2 \\
                f_3 \\
                \vdots \\
                f_n 
            \end{bmatrix}
            =
            \begin{bmatrix}
                const \\
                f_{2}(x_{1}) \\
                f_{3}(x_{1}, x_{2}) \\
                \vdots \\
                f_{n}(x_{1}, \ldots, x_{n-1})
            \end{bmatrix}
        \]
    \end{coloritemize}
    \footcitetext{nosov06abel}
\end{frame}


\section[Булев случай]{Эквивалентности в булевом случае}

\begin{frame}{Случай $k=2$}
    Посмотрим несколько (не все!) эквивалентных определений для случая $k=2$:
    \begin{coloritemize}
        \item одностоковые ориентации булевых кубов,
        \item булевы сети с наследственно единственной неподвижной точкой,
        \item отображение с несамодвойственными проекциями.
    \end{coloritemize}
\end{frame}


\begin{frame}{Одностоковые ориентации ($\uso$)}
    \begin{myexample}{Булев куб  $\BB_n$}
        \begin{coloritemize}
            \item вершины: $V = \{ \alpha \in \EE_2^n \}$;
            \item ребра: $\{\alpha, \beta \} \in E \Leftrightarrow \rho(\alpha, \beta) = 1$ (расстояние Хэмминга).
            \footcitetext{yablonski}
        \end{coloritemize}
    \end{myexample}
    \pause
    \begin{myexample}{Ориентация с единственным стоком $\uso$}
        \textbf{Ориентация с единственным стоком} (unique sink orientation, $\uso$) куба $\BB_n$~--- ориентированный граф, построенный по $\BB_n$ со следующим характеристическим свойством: в каждом подкубе $\BB_n$ существует единственный сток.
        \footcitetext{szabo2001, USOphd}
    \end{myexample}
    Такие геометрические конструкции исследовались в связи с задачами оптимизации.
\end{frame}  

  
\begin{frame}{$\uso$: один пример}
    \begin{figure}
        \centering
        \includegraphics[width=0.45\linewidth]{USO.png}
        \caption{Одностоковая ориентация трехмерного булева куба $\BB_3$}
    \end{figure}
\end{frame}


\begin{frame}%{Граф семейства $\Gamma(F)$}
    Пусть $\ff$~--- семейство булевых функций.
    \begin{myexample}{Граф семейства $\Gamma_{\ff}$}
        \begin{coloritemize}
            \item Вершины: $V = \{ \alpha \in \EE_2^n \}$.
            \item Пусть $\alpha \ne \beta$, $\rho(\alpha, \beta) = 1$, $\alpha_i \ne \beta_i$, добавим ориентированное ребро $(\beta, \alpha) \in E$ тогда и только тогда, когда $f_i(\alpha) = \alpha_i$.
        \end{coloritemize}
    \end{myexample}
    \begin{figure}
        \centering
        \includegraphics[width=0.45\linewidth]{edge.png}
    \end{figure}
\end{frame}


\begin{frame}{Стоки графа $\Gamma_{\ff}$}
    \begin{mytheorem}{Неподвижные точки булевых семейств}
        У правильного семейства булевых функций всегда существует единственная неподвижная точка.
    \end{mytheorem}
    \pause 
    \begin{coloritemize}
        \item Неподвижная точка $\alpha$ отображения $x \to \ff(x)$ соответствует стоку в графе $\Gamma_{\ff}$: 
        \(
            f_i(\alpha) = \alpha_i, \quad 1 \le i \le n.
        \)
        \pause 
        \item Ориентации подкубов в $\Gamma_{\ff}$ задаются проекциями $\ff'$ семейства $\ff$.
    \end{coloritemize}
    \begin{mytheorem}{Взаимно-однозначное соответствие}
        Граф $\Gamma_{\ff}$ семейства булевых функций $\ff$ является одностоковой ориентацией ($\uso$) тогда и только тогда, когда $\ff$~--- правильное семейство.
    \end{mytheorem}
\end{frame}


\begin{frame}{$\uso$ и правильность: два описания одного объекта}  
    \begin{coloritemize}
        \item Существует взаимно-однозначное соответствие между \textquote{алгебраическим} (определение правильности) и \textquote{геометрическим} ($\uso$) описаниями.
        \pause 
        \item Это позволяет переводить результаты с одного \textquote{языка} на другой (в частности, с помощью геометрической интуиции).
        \pause 
        \item Некоторые примеры переноса: вероятностный алгоритм порождения правильных семейств с помощью процедуры MCMC~\footcite{USOphd, galatenko21generation}, оценка на число булевых правильных семейств~\footcite{dm21}, новые классы правильных семейств: рекурсивно треугольные семейства, локально треугольные семейства.
    \end{coloritemize}    
\end{frame}


\begin{frame}{Пример переноса}
    Пусть $T(n)$ ($\Delta(n)$)~--- число \textbf{булевых} правильных (треугольных) семейств размера~$n$.
    \begin{mypropos}{Оценка на число булевых правильных семейств}
        \[
            n^{A \cdot 2^n} \le T(n) \le n^{B \cdot 2^n},
        \]
        где $A$, $B$~--- некоторые положительные константы.
        \footcitetext{numberUSO}
    \end{mypropos}
    \pause 
    \begin{mytheorem}{Булевых треугольных семейств экспоненциально мало}
        \[
            \frac{\Delta(n)}{T(n)} = o \Big(\frac{1}{n^{D \cdot 2^n}} \Big) \text{ при } n \to \infty,
        \]
        для некоторого $D > 0$.
        Почти все булевы правильные семейства не являются треугольными.
    \end{mytheorem}
\end{frame}


% \begin{frame}{Рекурсивная треугольность}
%     \begin{myexample}{Рекурсивно треугольное семейство}
%         Семейство $\ff \colon \EE_k^n \to \EE_k^n$ со свойством: существует $i$, такое что $f_i \equiv const_i$, и $\proj_a^i(\ff)$ рекурсивно треугольны для всех $a \in \EE_k$.
%     \end{myexample}
%     Обобщение понятия рекурсивной ориентации булева куба~\footnote{\cite[A141770]{oeis}}.
%     \pause
%     Треугольные семейства являются рекурсивно треугольными.
%     \pause
%     \begin{mytheorem}{О правильности рекурсивно треугольных семейств}
%         Рекурсивно треугольные семейства являются правильными.
%     \end{mytheorem}
%     \pause 
%     Теорема доказывается через локально треугольные семейства, о них позднее.
% \end{frame}


% \begin{frame}{Рекуррентное соотношение}
%     \begin{mytheorem}{Рекурсивно треугольных семейств мало}
%         Пусть $\Delta^{\rec}_k(n)$~--- число рекурсивно треугольных семейств размера~$n$ над $k$-значной логикой.
%         Тогда выполняется равенство:
%         \[
%             \Delta^{\rec}_k(n) = \sum_{j=1}^{n} (-1)^{j+1} \cdot k^j \cdot {n \choose j} \Delta^{\rec}_k(n-j)^{k^j}.
%         \]

%         Доля булевых рекурсивно треугольных семейств размера $n$ в классе всех булевых правильных семейств размера $n$ стремится к 0 при $n \to \infty$.
%     \end{mytheorem}
%     Обобщение результата для рекурсивной ориентации булева куба~\footnote{\cite[A141770]{oeis}}.
% \end{frame}


\begin{frame}{Неподвижные точки правильного семейства}
    \begin{mytheorem}{О неподвижных точках}
        Булево семейство $\ff$ является правильным тогда и только тогда, когда семейство $\ff$ и каждая из его проекций имеет единственную неподвижную точку.
    \end{mytheorem}
    \pause 

    Критерий имеет обобщение~\footcite{galatenko21criterion} на случай $k>2$, но формулируется чуть более сложно.

    %В общем случае: семейство $\ff \colon \EE_k^n \to \EE_k^n$ является правильным тогда и только тогда, когда для любой перекодировки $\ff$ все её проекции имеют единственную неподвижную точку~\footcite{galatenko21criterion}.

    \pause 
    В булевом случае свойство единственности неподвижной точки дает ещё одно характеристическое свойство правильных семейств, которое изучалось в контексте математической биологии (экспрессия генов~\footcite{thomas1991regulatory, richard2015fixed, ruet2015asynchronous, ruet2016local}).
\end{frame}


\begin{frame}%{$\hupf$-сети}
    \begin{myexample}{Булевы сети с наследственно единственной неподвижной точкой}
        $\hupf$-сеть (сеть с наследственно единственной неподвижной точкой, hereditarily unique fixed point network)~--- булево семейство $\ff$ со следующим свойством: $\ff$ и все его проекции имеют единственную неподвижную точку (как отображения $\EE_2^n \to \EE_2^n$).
        \footcitetext{richard2015fixed}
    \end{myexample}
    \pause 
    \begin{mytheorem}{Правильные семейства $\leftrightarrow$ $\hupf$-сети}
        Булево семейство $\ff$ является правильным $\Leftrightarrow$ $\ff$ задает $\hupf$-сеть. 
    \end{mytheorem}
    \pause
    Соответствие между булевыми правильными семействами и $\hupf$-сетями позволяет перенести (и обобщить) часть результатов, полученных в контексте изучения динамики таких сетей, на правильные семейства.
\end{frame}


% \begin{frame}
%     Пусть $\ff$~--- семейство размера $n$.
%     \begin{myexample}{{Локальный граф взаимодействий $G(\ff, \alpha)$}}
%         \begin{coloritemize}
%         \item Вершины: $V = \{1, \ldots, n\}$.
%         \item Ребра: $i \to j$ тогда и только тогда, когда $f_j$ существенно зависит от $x_i$  \textquote{локально} в точке $a$:
%         \[
%             f_j(\alpha_1, \ldots, \alpha_i, \ldots, \alpha_n) \ne f_j(\alpha_1, \ldots, \alpha_i \oplus 1, \ldots, \alpha_n).
%         \]
%         \end{coloritemize}
%     \end{myexample}
%     \pause 
%     \begin{mypropos}{Ациклические локальные графы}
%         Пусть $G(\ff, \alpha)$~--- ациклический для каждой точки $\alpha \in \EE_2^n$, тогда $\ff$~--- $\hupf$-сеть.
%         \footcitetext{shih2005combinatorial}
%     \end{mypropos}
% \end{frame}


% \begin{frame}{Локальный граф взаимодействий-2}
%     \begin{myexample}{Локально треугольные семейства}
%         $\ff \colon \EE_k^n \to \EE_k^n $ локально треугольно, если $G(\ff, \alpha)$ ацикличен для каждой точки $\alpha \in \EE_k^n$, где локальная зависимость $f$ от $x_i$ в точке $\alpha$ определяется неравенством:
%         \[
%             \exists b \colon f(\alpha_1, \ldots, \alpha_i, \ldots, \alpha_n) \ne f(\alpha_1, \ldots, b, \ldots, \alpha_n).
%         \]
%     \end{myexample}
%     \pause 
%     Локально треугольные семейства являются правильными~--- обобщение результата работы~\cite{shih2005combinatorial} на $k$-значную логику.
% \end{frame}


\begin{frame}{Характеризация через несамодвойственные проекции}
    \begin{myexample}{Самодвойственное семейство}
        Отображение $\ff \colon \EE_2^n \to \EE_2^k$ самодвойственно, если для любого набора $x \in \EE_2^n$ выполняется свойство $\ff(\overline{x}) = \overline{\ff(x)}$.
    \end{myexample}
    \pause 
    \begin{mytheorem}{О несамодвойственности проекций} 
        Семейство $\ff$ булевых функций правильно тогда и только тогда, когда каждая из его проекций 
        \[
            \proj^{a_1, \ldots, a_k}_{i_1, \ldots, i_k}(\ff)
        \] 
        \textbf{не является} самодвойственным булевым отображением.
    \end{mytheorem}
    По сути~--- следствие результата из~\cite{richard2015fixed}.
\end{frame}


\begin{frame}{Сложность распознавания правильности}
    \begin{coloritemize}
        \item В общем случае проверка правильности является сложной задачей: если семейство задано в форме КНФ, то задача проверки правильности coNP-полна~\footcite{nosov98}.
        \pause 
        \item В определенных случаях задача проверки правильности может быть упрощена, в частности, за счет вида графа существенной зависимости~\footcite{rykov14}.
        \pause 
        \item Алгоритм проверки правильности булева семейства требует порядка $\Theta(4^n)$ операций вычисления правильного семейства на двоичном наборе $x$ (проверка по определению правильности).
        \pause 
        \item За счет критерия несамодвойственности можно предложить адаптацию алгоритма~\footcite{bosshard2017pseudo} со сложностью $\Theta(3^n)$, проверяющего, что ориентация $\Gamma_{\ff}$, задаваемая семейством $\ff$, является одностоковой.
    \end{coloritemize}
\end{frame}



\section[$k$-значная логика]{Эквивалентности в случае $k$-значной логики}

\begin{frame}{Случай $k \ge 2$}
    \begin{coloritemize}
        \item Регулярные отображения.
        \item Клики в графах особого вида.
    \end{coloritemize}
\end{frame}

\begin{frame}
    \begin{mytheorem}{Критерий в терминах регулярности}
    \label{thm:regularity}
        Семейство $\ff_n$ на $\QQQ$ является правильным тогда и только тогда, когда для любого набора отображений 
        \(
            \psi_i \colon Q_i \to Q_i, \; 1 \le i \le n,
        \)
        следующее отображение из $\QQQ$ в себя биективно:
        \[
            \xx = 
            \begin{bmatrix}
                x_1\\
                \vdots \\
                x_n \\
            \end{bmatrix} 
            \to
            \xx \circ \Psi(\ff_n(\xx))
            = 
            \begin{bmatrix}
                x_1 \circ_1 \psi_1(f_1(x_1, \ldots, x_n)) \\
                \vdots \\
                x_n \circ_n \psi_n(f_n(x_1, \ldots, x_n))
            \end{bmatrix}, \; x_i \in Q_i.
        \]
    \end{mytheorem}
    Критерий обобщает известный результат~\footcite{nosov06abel} для абелевых групп.
\end{frame}


\begin{frame}{Случай $\FF_p$}
    \begin{mytheorem}{Регулярность для простых полей}
        Семейство функций $\ff$ на $\FF_p^n$, где $\FF_p$~--- простое поле, является правильным тогда и только тогда, когда для любых $a_i \in \FF_p$ отображение
        \[
            \begin{bmatrix}
                x_1\\
                \vdots \\
                x_n \\
            \end{bmatrix} 
            \to
            \begin{bmatrix}
                x_1 + a_1 \cdot f_1(x_1, \ldots, x_n)\\
                \vdots \\
                x_n + a_n \cdot f_n(x_1, \ldots, x_n)
            \end{bmatrix}, \; x_i \in \FF_p
        \]
        является биекцией $\FF_p^n \to \FF_p^n$.
    \end{mytheorem}
\end{frame}


\begin{frame}{Кликовое представление правильных семейств}
    \begin{coloritemize}
        \item Правильные семейства находятся во взаимно-однозначном соответствии с кликами некоторым образом построенного графа (\textquote{обобщенный граф Келлера}).
        \pause
        \item Для $k=2$ перенос из теории $\mathsf{USO}$-ориентаций~\footcite{borzechowski2022universal}, для $k > 2$~--- авторское обобщение.
        \pause
        \item Обобщенный граф Келлера $G(k, n)$: $V = \EE_{k^2}^n$, 
        \[
            \{v, w\} \in E \leftrightarrow \exists i, \, 1 \le i \le n \colon v_i \equiv w_i \text{ mod } \; k, \; v_i \ne w_i.
        \]
        \pause
        \item Графы примечательны тем, что в случае $k = 2$ некоторым образом кодируют неэквивалентные замощения пространства гиперкубами~\footcite{sikiric2007cube, mathew2013enumerating}.
    \end{coloritemize}
    \pause
    \begin{mytheorem}{Клики и правильные семейства}
        Каждой клике на $k^n$ вершинах в графе $G(k, n)$ можно поставить в биективное соответствие некоторое правильное семейство $\ff_n$ размера $n$ на $\EE_k^n$.
    \end{mytheorem}
\end{frame}



\begin{frame}
    \begin{center}
        {\Huge Спасибо за внимание!}
    \end{center}
\end{frame}
