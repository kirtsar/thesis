%!TEX root = ./pres.tex

\section{Введение}

\begin{frame}{Актуальность темы исследования}
    \begin{coloritemize}
        \item \textbf{Симметричная криптография}\footcite{otte2019gage, gligoroski2019s, edon80, mileva2009quasigroup, dimitrova2004quasigroup, markovski2005unbiased}: совершенные шифры, хэш-функции, поточные шифры, генераторы псевдослучайных чисел.
        \item \textbf{Асимметричная криптография}\footcite{gligoroski2008multivariate, gligoroski2011mqq, katyshev18, gribov15, markov20}: постквантовые схемы электронной подписи, схемы выработки общего ключа, гомоморфное шифрование.
        \item Приложения в \textbf{теории кодирования}~\footcite{nechaev98, nechaev04, markov2020nonassociative}.
        \item \textbf{Схемы аутентификации} и многое другое~\footcite{glukhov, artamonov18, shcherbacov2017elements, chauhan2021quasigroups}.
    \end{coloritemize}
\end{frame}


\begin{frame}
    \begin{myexample}{Квазигруппа}
        Пара $(Q, \circ)$, $Q$~--- (конечное) множество, $\circ \colon Q \times Q \to Q$,  для любых $a, b \in Q$ существуют единственные $x, y \in Q$, такие что: $a \circ x = b$, $y \circ a = b$.
        \footcitetext{belousov, keedwell}
    \end{myexample}
    \begin{coloritemize}
        \item В общем случае $Q$ задается таблицей размера $\lvert Q \rvert \times \lvert Q \rvert$.
        \item Случайная генерация (поиск + отсев) квазигрупп из некоторого узкого класса\footcite{gligoroski2008public, chen2010multivariate}.
        \item Итеративное построение из более \textquote{маленьких} (произведения)\footcite{gribovphd, EdonRprime}.
        \item Изотопы некоторых \textquote{хорошо изученных} групп (например, группы точек эллиптической кривой\footcite{DH16}).
        \item Функциональное задание квазигруппы: 
        \[
            x \circ y = z \leftrightarrow z_i = f_i(x_1, \ldots, x_n, y_1, \ldots, y_n). 
        \]
    \end{coloritemize}
\end{frame}


\begin{frame}{Криптографически релевантные свойства квазигрупп}
    \begin{coloritemize}
        \item Малое число ассоциативных троек, то есть троек элементов $(a, b, c) \in Q^3$
        \[
            (a \circ b) \circ c = a \circ (b \circ c)
        \]
        \item Отсутствие подквазигрупп, т.е. подмножеств $Q' \subset Q$, которые замкнуты относительно умножения.
        \item Полиномиальная полнота квазигрупп (любое отображение $f \colon Q^n \to Q$ задается с помощью композиции констант и операции умножения).
    \end{coloritemize}
\end{frame}


\begin{frame}{Используемые обозначения}
    \begin{table}
        \begin{center}
            \begin{tabular}{|c|c|}
                \hline
                $Q$ &квазигруппа с операцией $\circ$ \\
                \hline 
                $k$ & размер множества $Q$, $k = \lvert Q \rvert$, значность логики \\
                \hline
                $\EE_k$ & множество $\{0, \ldots, k-1 \}$ (обычно предполагаем $\EE_k = \ZZ_k$) \\
                \hline
                $\ff$ & семейство (набор) функций $\ff = (f_1, \ldots, f_n)$, \\
                      & $\ff \colon Q^n \to Q^n$ \\
                \hline 
                $f_i$ & $i$-я функция семейства $\ff$ \\
                \hline
                $n$ & размер семейства \\
                \hline
                $Func(Q)$ & множество функций $f \colon Q \to Q$ \\
                \hline 
                $Perm(Q)$ & множество подстановок (биекций) на $Q$ \\
                \hline
            \end{tabular}
        \end{center}
    \end{table}
\end{frame}


\begin{frame}
    \begin{myexample}{Правильное семейство}
        Семейство функций на $Q^n$ называется правильным, если для любых двух наборов $x \ne y$ найдется такая координата $i$, что $x_i \ne y_i$, но $f_i(x) = f_i(y)$.
        \footcitetext{nosov98}
    \end{myexample}

    \begin{coloritemize}
        \item Семейство булевых функций $\ff = (f_1, \ldots, f_n)$ является правильным тогда и только тогда\footcite{nosov99, nosov06abel}, когда отображение $(x, y) \to z = x \oplus y \oplus \ff(\pi_1(x_1, y_1), \ldots, \pi_n(x_n, y_n))$ задает квазигрупповую операцию \textbf{при любом выборе} внутренних функций $\pi_1, \ldots, \pi_n$.

        \item Пусть $\ff \colon \EE_k^n \to \EE_k^n$~--- правильное семейство, $M$~--- мощность образа отображения $x \to \ff(x)$.
        Тогда\footcite{galatenko23} число различных квазигрупп, порождаемых указанной конструкцией, не менее чем $M^{k^2}$.
    \end{coloritemize}
\end{frame}


\begin{frame}{Цели и задачи исследования}
    \textbf{Цель исследования:} изучение свойств правильных семейств функций, а также алгебраических свойств квазигрупп, заданных правильными семействами функций.

    \textbf{Задачи исследования}
    \begin{coloritemize}
        \item Получение новых критериев правильности семейств функций, а также установление естественного соответствия между правильными семействами функций и другими комбинаторно-алгебраическими структурами.
        \item Исследование общих свойств правильных семейств функций, включая структуру множества неподвижных точек, а также стабилизатор относительно определенных классов преобразований.
        \item Нахождение новых классов правильных семейств и изучение их свойств, включая мощность класса и мощность образа представителей.
        \item Разработка нового способа построения квазигрупп на основе правильных семейств функций, создание шифра, сохраняющего формат, на основе этой конструкции, и анализ характеристик полученного шифра.
    \end{coloritemize}
\end{frame}

\section{Глава 1: основные определения и примеры}


\begin{frame}
    \begin{mytheorem}{Критерий в терминах регулярности, \thm~1}
    \label{thm:regularity}
        Семейство $\ff_n$ на $\QQQ$ является правильным тогда и только тогда, когда для любого набора отображений 
        \(
            \psi_i \colon Q_i \to Q_i, \; 1 \le i \le n,
        \)
        следующее отображение из $\QQQ$ в себя биективно:
        \[
            \xx = 
            \begin{bmatrix}
                x_1\\
                \vdots \\
                x_n \\
            \end{bmatrix} 
            \to
            \xx \circ \Psi(\ff_n(\xx))
            = 
            \begin{bmatrix}
                x_1 \circ_1 \psi_1(f_1(x_1, \ldots, x_n)) \\
                \vdots \\
                x_n \circ_n \psi_n(f_n(x_1, \ldots, x_n))
            \end{bmatrix}, \; x_i \in Q_i.
        \]
    \end{mytheorem}
    Критерий обобщает известный результат~\footcite{nosov06abel} для абелевых групп.
\end{frame}


\begin{frame}{Один способ задания квазигруппы}
    Пусть $\ff$, $\gf$~--- два правильных семейства функций размера $n$ над группой $(\GGG^n, +)$.
    Для $\xx, \yy \in \GGG^n$ зададим операцию $\circ$ следующим образом:
    \[
        \xx \circ \yy = \xx + \ff(\xx) + \yy + \gf(\yy).
    \]
    \begin{coloritemize}
        \item Операция $\circ$ является квазигрупповой (\thm~1).
        \item Индексы ассоциативности квазигрупп, построенных по паре $(\ff, \gf)$ и по паре $(\gf, \ff)$, совпадают (\thm~5).
        \item Для $G = \ZZ_2$ индексы ассоциативности квазигрупп, построенных по паре $(\ff, \gf)$ и по паре $(\ff \oplus \alpha, \gf \oplus \alpha)$, совпадают (\thm~7).
        \item Для $G = \ZZ_2$ количество ассоциативных троек в квазигруппе, построенной по паре правильных булевых семейств $(\ff, \gf)$, четно (\thm~8).
    \end{coloritemize}
\end{frame}


\begin{frame}{Примеры правильных семейств}
    \begin{coloritemize}
        \item Константные семейства\footcite{nosov06abel} $f_i \equiv const_i$.
        \item Треугольные семейства $f_i = f_i(x_1, \ldots, x_{i-1})$.
        \item Класс квадратичных семейств (\thm~2):
        \[
            \ff(x_1, \ldots, x_n) = 
            \begin{bmatrix}
                0 \\
                x_1 \\
                x_1 \oplus x_2 \\
                \vdots \\
                x_1 \oplus x_2 \oplus \ldots \oplus x_{n-1}
                \end{bmatrix}
                \bigoplus
                \begin{bmatrix}
                \bigoplus_{i < j, \; i, j \ne 1}^n \; x_i x_j \\
                \bigoplus_{i < j, \; i, j \ne 2}^n \; x_i x_j \\
                \bigoplus_{i < j, \; i, j \ne 3}^n \; x_i x_j \\
                \vdots \\
                \bigoplus_{i < j, \; i, j \ne n}^n \; x_i x_j \\
            \end{bmatrix}.
        \]
    \end{coloritemize}
\end{frame}


\section{Глава 2: эквивалентные условия правильности семейств}


\begin{frame}
    Пусть $\ff$~--- правильное семейство.

    \begin{coloritemize}
        \item \textbf{Булев куб} $\BB_n$: вершины $V = \{ \alpha \in \EE_2^n \}$; ребра $\{\alpha, \beta \} \in E \Leftrightarrow \rho(\alpha, \beta) = 1$ (расстояние Хэмминга).
        \item \textbf{Граф семейства $\Gamma_{\ff}$:} Ориентируем ребро в графе $\EE_2^n$ следующим образом: добавим ориентированное ребро $(\beta, \alpha) \in E$ тогда и только тогда, когда $f_i(\alpha) = \alpha_i$.
    \end{coloritemize}

    \begin{columns}[T]
        \begin{column}{.4\textwidth}
            Неподвижная точка $\alpha$ отображения $x \to \ff(x)$ соответствует стоку в графе $\Gamma_{\ff}$
        \end{column}
        \hfill
        \begin{column}{.48\textwidth}
            \begin{figure}
                \centering
                \includegraphics[width=0.6\linewidth]{edge.png}
            \end{figure}
        \end{column}
    \end{columns}

    \begin{mytheorem}{\lem~1, \textbf{Следствие}~1}
        Булево семейство $\ff$ является правильным тогда и только тогда, когда семейство $\ff$ и каждая из его проекций имеет единственную неподвижную точку.
    \end{mytheorem}

    Критерий может быть обобщен на $k$-значную логику~\footcite{galatenko21criterion}.
\end{frame}


\begin{frame}
    \begin{columns}[T]
        \begin{column}{.5\textwidth}
            \textbf{Ориентация с единственным стоком} (unique sink orientation, $\uso$) куба $\BB_n$~--- ориентированный граф, построенный по $\BB_n$ со следующим характеристическим свойством: в каждом подкубе $\BB_n$ существует единственный сток.
            \footcitetext{szabo2001, USOphd}
        \end{column}
        \hfill
        \begin{column}{.5\textwidth}
            \begin{figure}
                \begin{center}
                    \includegraphics[width=0.5\linewidth]{USO.png}
                    \caption{Одностоковая ориентация трехмерного булева куба $\BB_3$}
                \end{center}
            \end{figure}
        \end{column}
    \end{columns}

    \begin{mytheorem}{Взаимно-однозначное соответствие, \thm~9}
        Граф $\Gamma_{\ff}$ семейства булевых функций $\ff$ является одностоковой ориентацией ($\uso$) тогда и только тогда, когда $\ff$~--- правильное семейство.
    \end{mytheorem}
\end{frame}  


\begin{frame}
    \begin{myexample}{Булевы сети с наследственно единственной неподвижной точкой}
        $\hupf$-сеть (сеть с наследственно единственной неподвижной точкой, hereditarily unique fixed point network)~--- булево семейство $\ff$ со следующим свойством: $\ff$ и все его проекции имеют единственную неподвижную точку (как отображения $\EE_2^n \to \EE_2^n$).
        \footcitetext{thomas1991regulatory, richard2015fixed, ruet2015asynchronous, ruet2016local}
    \end{myexample}

    \begin{mytheorem}{Правильные семейства $\leftrightarrow$ $\hupf$-сети, \thm~12}
        Булево семейство $\ff$ является правильным $\Leftrightarrow$ $\ff$ задает $\hupf$-сеть. 
    \end{mytheorem}

    Полученное соответствие (правильность $\Leftrightarrow$ $\uso$-ориентации, $\hupf$-сети) позволяет переводить результаты с одного \textquote{языка} на другой.
\end{frame}


\begin{frame}{Примеры переноса}
    \begin{coloritemize}
        \item Вероятностный алгоритм порождения правильных семейств с помощью процедуры MCMC~\footcite{USOphd, galatenko21generation}.
        \item Оценка на число $T(n)$ булевых правильных семейств
        \[
            \log_2 (T(n)) = \Theta \left (2^n \cdot \log_2 (n) \right ),
        \]
        оценка на долю треугольных правильных семейств среди всех правильных семейств (треугольных семейств экспоненциально мало, \thm~10).
        \item Новые классы правильных семейств: рекурсивно треугольные семейства (\lem~9), локально треугольные семейства (\thm~13).
        \item Характеризация через несамодвойственные проекции (\thm~14).
    \end{coloritemize} 
\end{frame}


\begin{frame}{Кликовое представление правильных семейств}
    \begin{coloritemize}
        \item Правильные семейства находятся во взаимно-однозначном соответствии с кликами некоторым образом построенного графа (\textquote{обобщенный граф Келлера}).
        \item Для $k=2$ перенос из теории $\mathsf{USO}$-ориентаций~\footcite{borzechowski2022universal}, для $k > 2$~--- авторское обобщение.
        \item Обобщенный граф Келлера $G(k, n)$: $V = \EE_{k^2}^n$, 
        \[
            \{v, w\} \in E \leftrightarrow \exists i, \, 1 \le i \le n \colon v_i \equiv w_i \text{ mod } \; k, \; v_i \ne w_i.
        \]
        \item Графы примечательны тем, что в случае $k = 2$ некоторым образом кодируют неэквивалентные замощения пространства гиперкубами~\footcite{sikiric2007cube, mathew2013enumerating}.
    \end{coloritemize}
    \begin{mytheorem}{\thm~15}
        Каждой клике на $k^n$ вершинах в графе $G(k, n)$ можно поставить в биективное соответствие некоторое правильное семейство $\ff_n$ размера $n$ на $\EE_k^n$.
    \end{mytheorem}
\end{frame}


\section{Глава 3: свойства правильных семейств}


\begin{frame}
    \begin{mypropos}{Преобразование перекодировки}
        \label{thm:reencoding}
        Пусть $\Phi, \Psi \in Perm(Q)^n$, $\Psi(x) = (\psi_1(x_1), \ldots, \psi_n(x_n))$ для $x \in Q^n$.
        Если $\ff$~--- правильное семейство, то $\Phi(\ff(\Psi(x)))$ также правильно.
        \footcitetext{galatenko21criterion}
    \end{mypropos}

    
    \begin{mypropos}{Согласованная перенумерация}
        Пусть $\sigma \in Perm(n)$, зададим преобразование $\ff \to \sigma(\ff)$ согласованной перенумерации следующим образом:
        \[
            f_i(x_1, \ldots, x_n) \to 
            f_{\sigma(i)}(x_{\sigma(1)}, \ldots, x_{\sigma(n)}).
        \]
        Если $\ff(x)$~--- правильное, то $\sigma(\ff)$ также правильно.
        \footcitetext{nosov06abel}
    \end{mypropos}
\end{frame}


\begin{frame}
    \begin{coloritemize}
        \item Согласованные перенумерации и перекодировки биективны, сохраняют правильность семейства, являются изометриями $\EE_k^n$ (в метрике Хэмминга).
        \item Общая постановка задачи: пусть $\Phi$, $\Psi$~--- биекции на $Q^n$: $\Phi, \Psi \in Perm(Q^n)$.
        Описать структуру стабилизатора множества всех правильных семейств, заданных на $Q^n$:
        \[
            \{(\Phi, \Psi) \in Perm(Q^n) \mid \Phi(\ff(\Psi(x))) \text{ правильно для любого правильного } \ff \colon Q^n \to Q^n \}.  
        \]
        
    \end{coloritemize}

    \begin{mytheorem}{Стабилизатор правильных семейств, \thm~19}
        Пусть семейства $\gf(\xx)$ вида $\gf(\xx) = \Phi(\ff(\Psi(\xx)))$ являются правильным для всех правильных семейств $\ff$, заданных на $\EE_k^n$, $\Phi$ и $\Psi$~--- биекции множества $\EE_k^n$.
        Тогда $\Phi$ и $\Psi$ имеют вид 
        \[
            \Phi = \sigma \circ A, \Psi = \sigma \circ B, 
        \]
        где $\sigma \in \SSS_n$~--- перенумерация, $A, B \in \left( \SSS_{\EE_k} \right)^n$~--- перекодировка.
    \end{mytheorem}
\end{frame}


\begin{frame}
    \begin{mypropos}{Ограниченность мощности образа, \propos~29}
        Число значений, принимаемых правильным семейством порядка~$n$ в $k$-значной логике, не превосходит~$k^{n-1}$.
        \footcitetext{galatenko23}
    \end{mypropos}
    \begin{mytheorem}{Мощность образа квадратичного семейства, \thm~21} 
        \[
            \begin{bmatrix}
            0 \\
            x_1 \\
            \vdots \\
            x_1 \oplus x_2 \oplus \ldots \oplus x_{n-1}
            \end{bmatrix}
            \bigoplus
            \begin{bmatrix}
            \bigoplus_{i < j, \; i, j \ne 1}^n \; x_i x_j \\
            \bigoplus_{i < j, \; i, j \ne 2}^n \; x_i x_j \\
            \vdots \\
            \bigoplus_{i < j, \; i, j \ne n}^n \; x_i x_j \\
            \end{bmatrix}
        \]
        Семейство имеет максимальную мощность образа~$2^{n-1}$.
    \end{mytheorem}
\end{frame}


\begin{frame}
    \[
        \ff_n(x) = 
        \begin{bmatrix}
            f_1(x_1, \ldots, x_n) \\
            f_2(x_1, \ldots, x_n) \\
            \vdots \\
            f_n(x_1, \ldots, x_n) \\
        \end{bmatrix}
        =
        \begin{bmatrix}
            \overline{x}_2 \cdot x_3 \\
            \overline{x}_3 \cdot x_4 \\
            \vdots \\
            \overline{x}_1 \cdot x_2 \\
        \end{bmatrix}.
    \]

    Семейство $\ff_n$ является правильным~\footcite{galatenko20quad} при нечетных $n$.
    
    \begin{mytheorem}{Мощность образа семейства, \thm~22}
        Мощность образа семейства $\ff_n$ равна $\lucas_n$ ($n$-е число Люка):
        \[
            \lucas_n = \lucas_{n-1} + \lucas_{n-2}, \quad \lucas_0 = 2, \quad \lucas_1 = 1.
        \]
    \end{mytheorem}
\end{frame}


\begin{frame}
    Пусть $\ff \colon Q^n \to Q^n$~--- правильное, $(Q, \circ)$~--- квазигруппа.
    Тогда отображение
    \[ 
        \sigma_\ff(x) \colon x \to x \circ \ff(x),
        \quad
        \begin{bmatrix}
            x_1 \\
            \vdots \\
            x_n
        \end{bmatrix} 
        \to 
        \begin{bmatrix}
            x_1 \circ f_1(x_1, \ldots, x_n) \\
            \vdots \\
            x_n \circ f_n(x_1, \ldots, x_n)
        \end{bmatrix}
    \]
    является подстановкой: $\sigma_{\ff} \in Perm(Q^n)$.

    \begin{mytheorem}{Обратимость \textquote{правильных подстановок}, \thm~23}
        Если $(Q, +)$~--- группа (т.е., операция $+$ ассоциативна), то семейство $\gf \colon Q^n \to Q^n$, определенное равенством
        \[
            \gf(x) = (-x) + \sigma_{\ff}^{-1}(x)
        \]
        также является правильным.
    \end{mytheorem}
\end{frame}


% \begin{frame}
%     Пусть $\ff \colon Q^n \to Q^n$~--- правильное.
%     Рассмотрим $\sigma^{-1}_{\ff} \in Perm(Q^n)$.
%     \begin{mytheorem}{Обратимость \textquote{правильных подстановок}, \thm~23}
%         Если $(Q, +)$~--- группа (т.е., операция $+$ ассоциативна), то семейство $\gf \colon Q^n \to Q^n$, определенное равенством
%         \[
%             \gf(x) = (-x) + \sigma_{\ff}^{-1}(x)
%         \]
%         также является правильным.
%     \end{mytheorem}
%     Если $\ff$~--- правильное, то существует правильное семейство $\gf$ со свойством
%     \[
%         \sigma^{-1}_{\ff}(x) = \sigma_{\gf}(x).
%     \]
%     Множество \textquote{правильных подстановок} замкнуто относительно взятия обратного элемента (в случае, когда $Q$~--- группа).
% \end{frame}


\begin{frame}{Подстановки, порождаемые правильными семействами}
    \begin{coloritemize}
        \item Множество \textquote{правильных подстановок} $\sprop$ замкнуто относительно взятия обратного элемента.
        \item $\sprop$ \textbf{не является} подгруппой $Perm(Q^n)$.
        \item $\langle \sprop \rangle$ действует транзитивно на $Q^n$ (любой элемент из $Q^n$ можно перевести в любой другой с помощью композиции некоторого количества $\sigma_{F}$).
        \item Если $Q = \EE_2$, то\footcite{USOphd} $\langle \sprop \rangle = Perm(\EE_2^n)$.
        \item Если $\ff$~--- правильное семейство булевых функций, то число решений уравнения $\ff(x) = \alpha$ четно для любого $\alpha \in \{0, 1\}^n$ (\thm~20); у подстановки $\sigma_{\ff}(x) = x \oplus {\ff}(x)$ чётное число неподвижных точек (\thm~24).
    \end{coloritemize}
\end{frame}


\section{Глава 4: алгоритмические и вычислительные аспекты}


\begin{frame}{Алгоритм шифрования, сохраняющего формат ($\fpe$-схема)}
    \begin{coloritemize}
        \item $\fpe$-схема~\footcite{bellare2009format}: алгоритм, позволяющий зашифровывать сообщения из произвольного конечного множества $M$ таким образом, что результат зашифрования также лежит в множестве $M$.
        \item Преобразуем $m \in M$, где $(M, \circ)$~--- квазигруппа, в $c \in M$ по правилу:
        \[
            m \to c = L_{k_1, \ldots, k_{\ell}}(m) = k_1 \circ \left( k_2 \circ ( \ldots ( k_{\ell} \circ m) \ldots ) \right).
        \]
        \item Элементы $k_i$ и последовательность сдвигов выбирается на основе мастер-ключа и настройки (tweak) псевдослучайным образом.
        \item Необходимо специфицировать конкретную квазигруппу.
    \end{coloritemize}
\end{frame}


\begin{frame}
    \begin{coloritemize}
        \item Пусть $\ff$, $\gf$~--- правильные семейства на $(H^n, +)$.
        \item Рассмотрим квазигруппу 
        \[
            (x, y) \to x \circ y = x + \ff(x) + y + \gf(y),
        \]
        \item Если $\ff$~--- правильное семейство на группе $H^n$, то семейство $\widetilde{\ff}$
        \[
            \widetilde{\ff}(x) = (-x) + \pi^{-1}_{\ff}(x), \quad \pi_{\ff}(x) = x + \ff(x), \quad x \in H^n,
        \]
        также является правильным на $H^n$.
        \item Таким образом, операция $x \circ y$ \textbf{обращается справа} следующим образом:
        \[
            x = \pi_{\widetilde{F}} \left( (x \circ y) - \pi_G(y) \right).
        \]
        \item Обращение слева также возможно.
        Обращение $\Leftrightarrow$ алгоритм расшифрования $\Leftrightarrow$ $\fpe$-схема.
    \end{coloritemize}
\end{frame}


\begin{frame}{Сложность распознавания правильности}
    \begin{coloritemize}
        \item В общем случае проверка правильности является сложной задачей: если семейство задано в форме КНФ, то задача проверки правильности coNP-полна~\footcite{nosov98}.
        \item В определенных случаях задача проверки правильности может быть упрощена, в частности, за счет вида графа существенной зависимости~\footcite{rykov14}.
        \item Алгоритм проверки правильности булева семейства требует порядка $\Theta(4^n)$ операций вычисления правильного семейства на двоичном наборе $x$ (проверка по определению правильности).
        \item Предложена адаптация алгоритма~\footcite{bosshard2017pseudo} со сложностью $\Theta(3^n)$, проверяющего, что ориентация $\Gamma_{\ff}$, задаваемая семейством $\ff$, является одностоковой.
        \item Алгоритм опирается на характеристическое свойство правильных семейств: булево семейство правильно тогда и только тогда, когда каждая его проекция не является самодвойственным отображением.
    \end{coloritemize}
\end{frame}


\begin{frame}{Численные эксперименты}
    \begin{coloritemize}
        \item Найдено точное число правильных, треугольных, рекурсивно/локально треугольных булевых семейств для малых значений $n \le 5$.
        \item Найдены индексы ассоциативности квазигрупп порядка $\lvert Q \rvert \in \{4, 8\}$, задаваемых парами правильных семейств, для $n = 16$ проведен статистический эксперимент; изучены свойства афинности и простоты.
    \end{coloritemize}
\end{frame}


\section{Заключение}


\begin{frame}{Положения, выносимые на защиту}
    \begin{coloritemize}
        \item Между булевыми правильными семействами и одностоковыми ориентациями графов булевых кубов ($\uso$-ориентациями), а также между булевыми правильными семействами и булевыми сетями с наследственно единственной неподвижной точкой ($\hupf$-сетями) существует естественное соответствие.
        Между правильными семействами в логике произвольной значности и кликами в обобщенных графах Келлера также существует естественное соответствие.
        \item Стабилизатор множества правильных семейств функций представляет собой множество пар согласованных изометрий пространства Хэмминга (согласованных перенумераций и перекодировок).
        \item Отображения, задаваемые правильными семействами булевых функций, всегда имеют четное число неподвижных точек.
    \end{coloritemize}
\end{frame}


\begin{frame}{Положения, выносимые на защиту}
    \begin{coloritemize}
        \item Мощность множества правильных семейств булевых функций размера $n$ $T(n)$ удовлетворяет отношению $\log_2 (T(n)) = \Theta \left (2^n \cdot \log_2 (n) \right )$.
        Треугольные семейства составляют бесконечно малую долю среди всех правильных семейств булевых функций.
    
        \item Локально треугольные, рекурсивно треугольные и сильно квадратичное семейства являются правильными. 
        Мощность образов рассмотренных в работе квадратичных булевых правильных семейств близка к максимально возможной.
        
        \item Предложенная в работе конструкция позволяет порождать квазигруппы с помощью правильных семейств функций. 
        Алгоритм шифрования, построенный на основе этой конструкции, сохраняет формат исходных сообщений (является $\fpe$-схемой).
        Ряд утверждений о числе ассоциативных троек в квазигруппах, построенных на основе предложенной конструкции, позволяет свести вопрос об изучении индексов ассоциативности от всех пар правильных семейств к классам эквивалентности пар правильных семейств.
    \end{coloritemize}
\end{frame}


\begin{frame}{Апробация работы}
    \begin{coloritemize}
        \item XXVI Международная конференция студентов, аспирантов и молодых учёных <<Ломоносов>>, Москва, Россия, с 8 по 12 апреля 2019~г.;

        \item X симпозиум <<Современные тенденции в криптографии>> (CTCrypt 2021), Дорохово, Россия, с 1 по 4 июня 2021~г.;

        \item XI симпозиум <<Современные тенденции в криптографии>> (CTCrypt 2022), Новосибирск, Россия, с 6 по 9 июня 2022~г.;

        \item Четырнадцатый международный семинар <<Дискретная математика и ее приложения>> имени академика О.Б. Лупанова под руководством В.~В.~Кочергина, Э.~Э.~Гасанова, С.~А.~Ложкина, А.~В.~Чашкина, с 20 по 25 июня 2022~г.;

        \item 11-я Международная конференция <<Дискретные модели в теории управляющих систем>>, Красновидово, Россия, с 26 по 29 мая 2023~г.;

        \item Третья Международная конференция ``MATHEMATICS IN ARMENIA: ADVANCES AND PERSPECTIVES'', Ереван, Армения, со 2 по 8 июля 2023~г.;
    \end{coloritemize}
\end{frame}


\begin{frame}{Апробация работы}
    \begin{coloritemize}
        \item 22-я Международная конференция <<Сибирская научная школа-семинар ``Компьютерная безопасность и криптография'' имени Геннадия Петровича Агибалова>>, Барнаул, Россия, с 4 по 9 сентября 2023~г.;

        \item Международная конференция <<Математика в созвездии наук>>, Москва, Россия, с 1 по 2 апреля 2024~г.;

        \item Международная конференция <<Алгебра и математическая логика: теория и приложения>>, Казань, Россия, с 27 июня по 1 июля 2024~г.;

        \item XX Международная научная конференция <<Проблемы теоретической кибернетики>>, Москва, Россия, с 5 по 8 декабря 2024~г.
    \end{coloritemize}
\end{frame}


\begin{frame}{Апробация работы}
    \begin{coloritemize}
        \item Научно-исследовательский семинар по алгебре механико-математического факультета МГУ под руководством Д.~О.~Орлова, М.~В.~Зайцева, 2023~г.

        \item Научно-исследовательский семинар <<Математические вопросы кибернетики>> кафедр дискретной математики и математической теории интеллектуальных систем механико-математического факультета и математической кибернетики факультета вычислительной математики и кибернетики МГУ под руководством Э.~Э.~Гасанова, В.~В.~Кочергина, С.~А.~Ложкина, 2023~г.

        \item Семинар <<Компьютерная алгебра>> факультета ВМК МГУ и ВЦ РАН под руководством профессора С.~А.~Абрамова, 2023~г.

        \item Семинар <<Теория автоматов>> механико-математического факультета МГУ под руководством профессора Э.~Э.~Гасанова, 2023~г.

        \item Семинар <<Современные проблемы криптографии>> под руководством ведущего научного сотрудника В.~А.~Носова и доцента А.~Е.~Панкратьева, механико-математический факультет МГУ, неоднократно.
    \end{coloritemize}
\end{frame}


\begin{frame}
    Основные результаты по теме диссертации изложены в~\textbf{9}~печатных изданиях,
    \textbf{8} из которых опубликованы в рецензируемых научных изданиях, рекомендованных для защиты в диссертационном совете МГУ по специальности 1.1.5. Математическая логика, алгебра, теория чисел и дискретная математика, из них \textbf{6} "--- в~ рецензируемых научных изданиях, входящих в ядро РИНЦ и международные базы цитирования (Web of Science / Scopus), RSCI, \textbf{2} "--- в~ рецензируемых научных изданиях из дополнительного списка МГУ, рекомендованных для защиты в диссертационном совете МГУ по специальности 1.1.5. Математическая логика, алгебра, теория чисел и дискретная математика и входящих в список ВАК.
\end{frame}


\begin{frame}{Публикации по теме диссертации}
    \begin{coloritemize}
        \item \textquote{О соответствии между правильными семействами и реберными ориентациями булевых кубов}, Интеллектуальные системы. Теория и приложения, 24:1 (2020), 97--100.
        \item \textquote{О взаимно однозначном соответствии между правильными семействами булевых функций и рёберными ориентациями булевых кубов}, ПДМ, 2020, 48,  16--21 (2020).
        \item \textquote{О свойствах правильных семейств булевых функций}, Дискрет. матем., 33:1 (2021), 91--102.
        \item ``Format-preserving encryption: a survey'', Матем. вопр. криптогр., 13:2 (2022),  133--153.
        \item \textquote{Об одном квазигрупповом алгоритме шифрования, сохраняющего формат}, ПДМ. Приложение, 2023, 16,  102--104.
        \item \textquote{Об индексе ассоциативности конечных квазигрупп}, Интеллектуальные системы. Теория и приложения, 28:3 (2024), 80--101. 
    \end{coloritemize}
\end{frame}


\begin{frame}{Публикации автора (в соавторстве)}
    \begin{coloritemize}
        \item A. V. Galatenko, V. A. Nosov, A. E. Pankratiev, K. D. Tsaregorodtsev, ``Proper families of functions and their applications'', Матем. вопр. криптогр., 14:2 (2023),  43--58.
        \item А. В. Галатенко, В. А. Носов, А. Е. Панкратьев, К. Д. Царегородцев, \textquote{О порождении n-квазигрупп с помощью правильных семейств функций}, Дискрет. матем., 35:1 (2023),  35--53.
        \item A. V. Galatenko, A. E. Pankratiev, K. D. Tsaregorodtsev,``A Criterion of Properness for a Family of Functions'', Journal of Mathematical Sciences, 284:4 (2024), 451--459.
    \end{coloritemize}
\end{frame}