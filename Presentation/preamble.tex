\begin{frame}[noframenumbering,plain]
    \setcounter{framenumber}{1}
    \maketitle
\end{frame}

\begin{frame}
    \frametitle{Положения, выносимые на защиту}
    \begin{itemize}
        \item Доказательство существования естественного соответствия между булевыми правильными семействами и одностоковыми ориентациями графов булевых кубов ($\uso$-ориентации), а также между булевыми правильными семействами и булевыми сетями с наследственно единственной неподвижной точкой ($\hupf$-сети).
        \item Доказательство существования естественного соответствия между правильными семействами в логике произвольной значности и кликами в обобщенных графах Келлера.
        \item Доказательство равенства стабилизатора множества правильных семейств функций и множества пар согласованных изометрий пространства Хэмминга (согласованные перенумерации и перекодировки).
        \item Доказательство того факта, что отображения, задаваемые с помощью правильных семейств булевых функций, всегда имеют четное число неподвижных точек.
    \end{itemize}
\end{frame}


\begin{frame}
    \frametitle{Положения, выносимые на защиту-2}
    \begin{itemize}
        \item Оценка на число правильных семейств булевых функций и оценка доли треугольных семейств среди всех правильных семейств булевых функций.
        \item Новые классы правильных семейств функций (рекурсивно треугольные, локально треугольные, сильно квадратичное семейство).
        \item Оценки на число рекурсивно треугольных семейств.
        \item Точные значения мощности образа отображений, задаваемых некоторыми обнаруженными примерами правильных семейств.
        \item Новый способ порождения квазигрупп на основе правильных семейств функций; доказательство ряда утверждений о числе ассоциативных троек в порождаемых квазигруппах.
        \item Новый алгоритм шифрования, сохраняющего формат ($\fpe$-схема), основанный на квазигрупповых операциях.
    \end{itemize}
\end{frame}

\begin{frame}
    \frametitle{Содержание}
    \tableofcontents
\end{frame}
