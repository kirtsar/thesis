\chapter*{Заключение}                       % Заголовок
\addcontentsline{toc}{chapter}{Заключение}  % Добавляем его в оглавление

%% Согласно ГОСТ Р 7.0.11-2011:
%% 5.3.3 В заключении диссертации излагают итоги выполненного исследования, рекомендации, перспективы дальнейшей разработки темы.
%% 9.2.3 В заключении автореферата диссертации излагают итоги данного исследования, рекомендации и перспективы дальнейшей разработки темы.
%% Поэтому имеет смысл сделать эту часть общей и загрузить из одного файла в автореферат и в диссертацию:

    В начале диссертационного исследования были поставлены следующие задачи.
    \begin{enumerate}
        \item Получение новых критериев правильности семейств функций, а также установление естественного соответствия между правильными семействами функций и другими комбинаторно-алгебраическими структурами.
        \item Исследование общих свойств правильных семейств функций, включая структуру множества неподвижных точек, а также стабилизатор относительно определенных классов преобразований.
        \item Нахождение новых классов правильных семейств и изучение их свойств, включая мощность класса и мощность образа представителей.
        \item Разработка нового способа построения квазигрупп на основе правильных семейств функций, создание шифра, сохраняющего формат, на основе этой конструкции, и анализ характеристик полученного шифра.
    \end{enumerate}
    Основные результаты работы заключаются в следующем.
    %% Согласно ГОСТ Р 7.0.11-2011:
%% 5.3.3 В заключении диссертации излагают итоги выполненного исследования, рекомендации, перспективы дальнейшей разработки темы.
%% 9.2.3 В заключении автореферата диссертации излагают итоги данного исследования, рекомендации и перспективы дальнейшей разработки темы.

    \begin{enumerate}
        \item Установлено естественное соответствие между булевыми правильными семействами и одностоковыми ориентациями графов булевых кубов ($\uso$-ориентации).
        \item Установлено естественное соответствие между булевыми правильными семействами и булевыми сетями с наследственно единственной неподвижной точкой ($\hupf$-сети).
        \item Установлено естественное соответствие между правильными семействами в логике произвольной значности и кликами в обобщенных графах Келлера.
        \item Доказано, что стабилизатором множества правильных семейств функций являются изометрии пространства Хэмминга (согласованные перенумерации и перекодировки).
        \item Показано, что отображения, задаваемые с помощью правильных семейств булевых функций, всегда имеют четное число неподвижных точек.
        \item Получена оценка на число правильных семейств булевых функций, предложены оценки доли треугольных семейств среди всех правильных семейств булевых функций.
        \item Обнаружены и исследованы новые классы правильных семейств функций (рекурсивно треугольные, локально треугольные, сильно квадратичное семейство).
        \item Получены оценки на число рекурсивно треугольных семейств.
        \item Для некоторых правильных семейств булевых функций получены точные значения мощности образа отображений, задаваемых этими правильными семействами.
        \item Предложен новый способ порождения квазигрупп на основе правильных семейств функций.
        \item Доказан ряд утверждений о числе ассоциативных троек в порождаемых квазигруппах.
        \item Предложен новый алгоритм шифрования, сохраняющего формат ($\fpe$-схема), основанный на квазигрупповых операциях.
    \end{enumerate}

    В качестве тем для дальнейших исследований можно отметить следующие направления.

    \begin{enumerate}
        \item Предложить способ построения достаточно широких классов правильных семейств с хорошими алгебраическими и комбинаторными свойствами, в том числе и для логик большей значности $k > 2$.

        \item Предложить способ быстрого построения множества представителей всех правильных семейств размера $n+1$ с помощью представителей размера $n$ и менее (с точностью до согласованных перенумераций и перекодировок).

        \item Предложить альтернативные геометрические описания правильных семейств в $k$-значной логике, где $k>2$, которые были бы инвариантны относительно согласованных перенумераций и перекодировок.

        \item Предложить алгоритм, полиномиальный по длине входа, на вход принимающий правильное семейство (например, в виде КНФ или полиномов Жегалкина) и параметрические подстановки и выдающий количество ассоциативных троек (или нижние и верхние границы на число троек), проверяющий полиномиальную полноту порождаемой квазигруппы, наличие или отсутствие подквазигрупп.

        \item Оценить генерическую сложность задачи решения системы уравнений над квазигруппами, заданными правильными семействами.
    \end{enumerate}

    Автор выражает глубокую благодарность своим научным руководителям А.~В.~Галатенко и А.~Е.~Панкратьеву за оказанную помощь при написании настоящей работы, постановку задачи, обсуждение результатов и постоянное внимание к работе.
