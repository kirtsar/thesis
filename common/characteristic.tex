% актуальность
{\actuality} 

    Диссертация посвящена вопросам, лежащим на стыке дискретной математики (теория дискретных функций), алгебры (теория квазигрупп) и криптографии.
    Основным объектом изучения является особый класс дискретных функций, введенных В. А. Носовым~\autocite{nosov98, nosov99} (т.н. <<правильные семейства>> функций), которые могут быть использованы для построения параметрических классов квазигрупп.
    %Свойства полученных квазигрупп связаны со свойствами используемого при их построении правильного семейства.
    Квазигруппы~--- одна из базовых структур в алгебре.
    Таблицы умножения квазигрупп, более известные под названием \textquote{латинские квадраты}, с древнейших времен и по настоящее время используются в различных областях математики (см., например, монографию Й. Денеша и Э.Д. Кидвелла~\autocite{keedwell}): при планировании статистических экспериментов, в играх и головоломках, а также в теории кодирования и криптографии, которые рассматриваются более подробно в настоящей работе.
    Из общих обзоров криптографических приложений квазигрупп можно отметить следующие источники:
    \begin{itemize}
        \item статья М.М. Глухова~\autocite{glukhov}, в которой приводятся примеры кодов аутентификации, шифров и однонаправленных функций на основе квазигрупповых преобразований, а также недавний обзор индийских авторов~\autocite{chauhan2021quasigroups}, затрагивающий тематику построения симметричных криптопримитивов на основе квазигрупповых операций;
        \item монография В. Щербакова~\autocite{shcherbacov2017elements}, в которой довольно подробно освещена тематика использования квазигрупп в криптографии; в частности, в работе рассматриваются следующие темы: поточные шифры и их криптоанализ, хэш-функции и односторонние функции, схемы разделения секрета; а также смежная тематика теории кодирования (в частности, рекурсивные МДР-коды);
        \item монография Й. Денеша и Э.Д. Кидвелла~\autocite{keedwell} и статья М.Э. Тужилина~\autocite{tuzhilin2012}, посвященные общим обзорам тематики латинских квадратов, их использованию в докомпьютерный этап развития криптографии и современным приложениям.
    \end{itemize}

    В качестве непосредственного приложения квазигрупп в области симметричной криптографии можно привести следующие механизмы, основанные на квазигрупповых операциях, предлагаемые к рассмотрению в статьях македонских авторов С. Марковски, Д. Глигороски, В. Димитровой, А. Милевой и т.д.:
    \begin{itemize}
        \item поточные шифры и хэш-функции, основанные на квазигрупповом умножении~\autocite{markovski1999quasigroup, markovski2003quasigroup, markovski2017quasigroup, snavsel2009hash},
        \item кандидат на стандартизацию в качестве поточного шифра \textbf{Edon80}~\autocite{edon80},
        \item кандидаты на стандартизацию в качестве хэш-функции \textbf{Edon-R}~\autocite{EdonR, EdonRprime} и \textbf{NaSHA}~\autocite{nasha, mileva2009quasigroup},
        \item кандидаты на стандартизацию в качестве низкоресурсной хэш-функции и алгоритма шифрования с ассоциированными (присоединенными) данными (AEAD-алгоритм) \textbf{GAGE} и \textbf{InGAGE}~\autocite{otte2019gage, gligoroski2019s},
        \item предложения Г. Теселеану~\autocite{tecseleanu2021quasigroups, tecseleanu2022security, tecseleanu2023cryptographic} и И.В. Чередника~\autocite{cherednik17, cherednik19, cherednik20} по использованию квазигрупповых операций в рамках (обобщенных) сетей Фейстеля.
    \end{itemize}
    Однако недостаточная изученность задач, лежащих в основании подобных предложений, иногда приводит к возможности довольно простого криптоанализа полученных решений (см. работы М. Войводы и И. Сламинковой~\autocite{vojvoda2004cryptanalysis,vojvoda2007note,slaminkova2010cryptanalysis}, М. Хелла и Т. Йохансона~\autocite{hell2007key}, И. Николича и Д. Ховратовича~\autocite{nikolicfree}, Ж. Ли и соавторов~\autocite{li2010collision}).
    %В качестве примеров применения квазигрупп в практических приложениях можно отметить следующие предложения:    

    Квазигруппы (а также более сложные алгебраические структуры, в основе которых лежат квазигруппы) и их приложения в теории кодирования исследовались в ряде работ за авторством С. Гонсалеса, Е. Коусело, В.Т. Маркова, А.А. Нечаева, А.В. Михалёва, А.В. Грибова и других.
    Так, в статье \autocite{nechaev98} исследуются $k$-рекурсивные коды (т.е. коды, для которых позиции в кодовых словах с номерами $i+k$ однозначно определяются по позициям $i, i+1, \ldots, i+k-1$ для $i = k+1, \ldots, n-k$, иначе говоря, $u_{i+k} = f(u_i, \ldots, u_{i+k-1})$), лежащие на границе Синглтона (МДР-коды).
    Подход, основанный на применении ортогональных латинских квадратов, позволяет получить в данном случае оценки на максимальную длину кодовых слов.
    В серии работ~\autocite{nechaev04, couselo2004loop, markov12, markov2020nonassociative} используются так называемые луповые кольца (формальные суммы квазигрупповых элементов) для построения различных оптимальных в разных смыслах кодов.

    Луповые кольца и другие алгебраические структуры, основанные на квазигруппах, могут быть использованы для построения множества асимметричных криптографических примитивов.
    Такие конструкции исследовались С.Ю. Катышевым, В.Т. Марковым, А.А. Нечаевым, А.В. Михалёвым, А.В. Барышниковым, А.В. Грибовым, А.В. Зязиным, Е.С. Кислициным и другими авторами.
    В качестве примера можно привести следующие криптографические схемы и протоколы:
    \begin{itemize}
        \item протоколы формирования общего ключа~--- аналоги протокола Диффи-Хеллмана~\autocite{DH14, DH16, baryshnikov2017key, quantum18};
        \item схемы асимметричного шифрования~\autocite{markov12, pke10, gribovphd};
        \item схемы гомоморфного шифрования~\autocite{gribovphd, homomo15, katyshev2020application, markov20}.
    \end{itemize}
    Отдельно можно выделить ряд работ, в которых изучаются схемы асимметричного шифрования и цифровой подписи, основанные на сложности решений систем уравнений в конечных полях (см. работы Д. Глигороски, С. Марковски, С. Кнапскога, Й. Ченя и других~\autocite{gligoroski2008public, gligoroski2008multivariate, chen2010multivariate, gligoroski2011mqq}).

    При этом применяемые в области защиты информации квазигруппы часто имеют довольно большие размеры (см., например, требования к квазигруппе в работах Д. Глигороски и соавторов~\autocite{EdonR, EdonRprime, chen2010multivariate}), что делает затруднительным поэлементное хранение в памяти компьютера всей таблицы умножения. 
    Так, например, для построения хэш-функции Edon-$\mathcal{R}'$ необходимо задать квазигруппу порядка $2^{256}$. 
    В связи с этим обстоятельством в большинстве предлагаемых криптосистем большая квазигруппа строится, как правило, согласно одному из следующих подходов:
    \begin{itemize}
        \item случайная генерация квазигруппы (случайных поиск подходящей квазигруппы совместно с процедурой отсева неподходящих) из некоторого узкого класса (Д. Глигороски и соавторы~\autocite{gligoroski2008public, chen2010multivariate});
        \item итеративное построение большой квазигруппы из квазигрупп меньшего размера (Д. Глигороски и соавторы~\autocite{EdonRprime}, А.В. Грибов~\autocite{gribovphd}) с помощью конструкций произведений;
        \item изотопы некоторых \textquote{хорошо изученных} групп: например, изотоп группы точек эллиптической кривой (В.Т. Марков, А.В. Михалёв, А.А. Нечаев~\autocite{DH16}), модульное вычитание (В. Снашель и соавторы~\autocite{snavsel2009hash});
        \item функциональное задание квазигруппы, рассматриваемое в настоящей работе более подробно.
    \end{itemize}

    В работах В.А. Носова~\autocite{nosov98, nosov99} был предложен метод задания латинского квадрата при помощи семейства булевых функций, которое определяет элемент квадрата по его координатам (номеру строки и столбца).
    Такие семейства функций, задающие целые параметрические классы латинских квадратов, были названы правильными.
    Понятие правильного семейства функций было сначала обобщено на случай абелевых групп (см. работы В.А. Носова, А.Е. Панкратьева, А.А. Козлова~\autocite{nosov06, nosov06abel, nosov07, nosov08, kozlov08}), а затем и на более общие алгебраические структуры (см. работы И.А. Плаксиной~\autocite{plaksina14} и А.В. Галатенко, В.А. Носова, А.Е. Панкратьева\autocite{galatenko2020latin}).
    Ряд работ посвящен изучению свойств введенных булевых отображений:
    \begin{itemize}
        \item В.А. Носовым~\autocite{nosov98} было (среди прочего) показано, что проверка свойства правильности является $\mathsf{coNP}$-полной задачей (т.е. в общем случае задача проверки правильности является сложной),
        \item в работах В.А. Носова, А.Е. Панкратьева, А.А. Козлова~\autocite{nosov07, nosov08, kozlov08} рассматривались свойства т.н. графа существенной зависимости правильных семейств (граф на $n$ вершинах, ребро $ i \to j$ присутствует в графе тогда и только тогда, когда $j$-я функция семейства зависит существенно от $x_i$) и были выделены широкие классы семейств, для которых свойство правильности эквивалентно свойству отсутствия циклов в графе существенной зависимости,
        \item в работах Д.О. Рыкова~\autocite{rykov10, rykov14} показано, как задача проверки свойства правильности может быть упрощена, если дополнительно известна структура графа существенной зависимости семейства,
        \item работы И.А. Плаксиной~\autocite{plaksina14} и А.В. Галатенко, В.А. Носова, А.Е. Панкратьева~\autocite{galatenko2020latin} посвящены, в том числе, различным способам задания \mbox{($d$-)квазигрупп} с помощью правильных семейств над различными алгебраическими структурами,
        \item работы А.В. Галатенко, В.А. Носова, А.Е. Панкратьева, В.М. Староверова~\autocite{galatenko21generation, galatenko2022generation} посвящены вопросам построения новых правильных семейств функций из старых.
    \end{itemize}

    При этом не всякая квазигруппа подходит для реализации на ее основе криптографических примитивов.
    Критически важными являются алгебраические свойства используемой квазигруппы, такие как свойства полиномиальной полноты (И. Хагеманн, К. Херрман~\autocite{hagemann}; Т. Нипков~\autocite{nipkow1990unification}; Г. Хорвац и соавторы~\autocite{horvath2008}; В.А. Артамонов и соавторы~\autocite{artamonov2016characterization}), количество ассоциативных троек (Т. Кепка~\autocite{kepka1980note}; А. Котзиг, К. Райшер~\autocite{kotzig83}; Ж. Жезек, Т. Кепка~\autocite{ass_summary}), наличие подквазигрупп (см., например, работу П.И. Собянина~\autocite{sobyanin19} и А.В. Галатенко, А.Е. Панкратьева, В.М. Староверова~\autocite{galatenko21subquasi}).
    В ряде работ изучаются свойства квазигрупп, порождаемых правильными семействами булевых функций:
    \begin{itemize}
        \item Н.А. Пивнем~\autocite{piven18} исследуются алгебраические свойства квазигрупп размера~4, порождаемых правильными семействами булевых функций размера $n = 2$, вводится понятие \textquote{перестановочной конструкции} (способ получения новых квазигрупп из уже имеющихся),
        \item в работе Н.А. Пивня~\autocite{piven19} рассмотрена избыточность \textquote{перестановочной конструкции} (различные значения параметров могут давать одну и ту же квазигруппу) и способы сокращения избыточности,
        \item в работе А.В. Галатенко, В.А. Носова, А.Е. Панкратьева~\autocite{galatenko20quad} предложен способ построения квадратичных квазигрупп, которые являются оптимальными с точки зрения криптографических приложений (обладают наиболее компактным представлением, при этом задача решения систем уравнений над подобными квазигруппами является в общем случае сложной),
        \item в дипломной работе А.С. Шварёва~\autocite{shvaryov24}, среди прочего, рассмотрены \textquote{криптографические} свойств квазигрупп, порождаемых правильными семействами (линейная, дифференциальная характеристики) и способы их \textquote{усиления}.
    \end{itemize}

    В контексте проведенных исследований остаются актуальными ряд нерешенных задач, исследованию которых и посвящена настоящая работа:
    \begin{itemize}
        \item изучение правильных семейств и их свойств как одного из возможных способов функционального задания квазигрупповой операции,
        \item изучение свойств квазигрупп, порождаемых правильными семействами.
    \end{itemize}

\ifsynopsis
%Этот абзац появляется только в~автореферате.
\else
% Этот абзац появляется только в~диссертации.
\fi

% % цель работы
{\aim} исследования является изучение свойств правильных семейств функций, а также алгебраических свойств квазигрупп, заданных правильными семействами функций.
Тема, объект и предмет диссертационной работы соответствуют следующим пунктам паспорта специальности 1.1.5. Математическая логика, алгебра, теория чисел и дискретная математика: теория алгебраических структур (полугрупп, групп, колец, полей, модулей и т.д.), теория дискретных функций и автоматов, теория графов и комбинаторика. 
%\TODO{теория сложности вычислений сюда плохо ложится, она относится к теоретической информатике.}


Для~достижения поставленной цели автору необходимо было решить следующие {\tasks}:
\begin{enumerate}[beginpenalty=10000] % https://tex.stackexchange.com/a/476052/104425
    \item Получение новых критериев правильности семейств функций, а также установление естественного соответствия между правильными семействами функций и другими комбинаторно-алгебраическими структурами.
    \item Исследование общих свойств правильных семейств функций, включая структуру множества неподвижных точек, а также стабилизатор относительно определенных классов преобразований.
    \item Нахождение новых классов правильных семейств и изучение их свойств, включая мощность класса и мощность образа представителей.
    \item Разработка нового способа построения квазигрупп на основе правильных семейств функций, создание шифра, сохраняющего формат, на основе этой конструкции, и анализ характеристик полученного шифра.
\end{enumerate}


% научная новизна
{\novelty}
результаты диссертации являются новыми и получены автором самостоятельно. 
Все результаты, выносимые автором на защиту, получены им лично. 
Результаты других авторов, используемые в диссертации, отмечены соответствующими ссылками.
Основные результаты диссертации состоят в следующем.
\begin{enumerate}[beginpenalty=10000] % https://tex.stackexchange.com/a/476052/104425
    \item Установлено естественное соответствие между булевыми правильными семействами и одностоковыми ориентациями графов булевых кубов ($\uso$-ориентации), а также между булевыми правильными семействами и булевыми сетями с наследственно единственной неподвижной точкой ($\hupf$-сети); установлено естественное соответствие между правильными семействами в логике произвольной значности и кликами в обобщенных графах Келлера.
    \item Доказано, что стабилизатором множества правильных семейств функций являются изометрии пространства Хэмминга (согласованные перенумерации и перекодировки); показано, что отображения, задаваемые с помощью правильных семейств булевых функций, всегда имеют четное число неподвижных точек; 
    получена оценка на число правильных семейств булевых функций, предложены оценки доли треугольных семейств среди всех правильных семейств булевых функций.
    \item Построены новые классы правильных семейств функций (рекурсивно треугольные, локально треугольные, сильно квадратичное семейство); получены оценки на число рекурсивно треугольных семейств; для некоторых правильных семейств булевых функций получены точные значения мощности образа отображений, задаваемых этими правильными семействами.
    \item Предложен новый способ порождения квазигрупп на основе правильных семейств функций; доказан ряд утверждений о числе ассоциативных троек в порождаемых квазигруппах; предложен новый алгоритм шифрования, сохраняющего формат ($\fpe$-схема), основанный на квазигрупповых операциях.
\end{enumerate}

%{\influence} \ldots

{\methods} В работе используются методы алгебры, дискретной математики, криптографии, теории графов, теории сложности.

{\defpositions}
\begin{enumerate}[beginpenalty=10000] % https://tex.stackexchange.com/a/476052/104425
    \item Между булевыми правильными семействами и одностоковыми ориентациями графов булевых кубов ($\uso$-ориентациями), а также между булевыми правильными семействами и булевыми сетями с наследственно единственной неподвижной точкой ($\hupf$-сетями) существует естественное соответствие.
    Между правильными семействами в логике произвольной значности и кликами в обобщенных графах Келлера также существует естественное соответствие.
    %\item Существование естественного соответствия между булевыми правильными семействами и одностоковыми ориентациями графов булевых кубов ($\uso$-ориентации), а также между булевыми правильными семействами и булевыми сетями с наследственно единственной неподвижной точкой ($\hupf$-сети) и его доказательство; существование естественного соответствия между правильными семействами в логике произвольной значности и кликами в обобщенных графах Келлера и его доказательство.
    \item Стабилизатор множества правильных семейств функций представляет собой множество пар согласованных изометрий пространства Хэмминга (согласованных перенумераций и перекодировок).
    \item Отображения, задаваемые правильными семействами булевых функций, всегда имеют четное число неподвижных точек.
    \item Мощность множества правильных семейств булевых функций размера $n$ $T(n)$ удовлетворяет отношению $\log_2 (T(n)) = \Theta \left (2^n \cdot \log_2 (n) \right )$.
    Треугольные семейства составляют бесконечно малую долю среди всех правильных семейств булевых функций.
    %\item Равенство стабилизатора множества правильных семейств функций и множества пар согласованных изометрий пространства Хэмминга (согласованные перенумерации и перекодировки); доказательство того факта, что отображения, задаваемые с помощью правильных семейств булевых функций, всегда имеют четное число неподвижных точек; оценка на число правильных семейств булевых функций и оценка доли треугольных семейств среди всех правильных семейств булевых функций.
    \item Локально треугольные, рекурсивно треугольные и сильно квадратичное семейства являются правильными. 
    Мощность образов рассмотренных в работе квадратичных булевых правильных семейств близка к максимально возможной.
    %\item Новые классы правильных семейств функций (рекурсивно треугольные, локально треугольные, сильно квадратичное семейство); оценки на число рекурсивно треугольных семейств; точные значения мощности образа отображений, задаваемых некоторыми обнаруженными примерами правильных семейств.
    \item Предложенная в работе конструкция позволяет порождать квазигруппы с помощью правильных семейств функций. 
    Алгоритм шифрования, построенный на основе этой конструкции, сохраняет формат исходных сообщений (является $\fpe$-схемой).
    Ряд утверждений о числе ассоциативных троек в квазигруппах, построенных на основе предложенной конструкции, позволяет свести вопрос об изучении индексов ассоциативности от всех пар правильных семейств к классам эквивалентности пар правильных семейств.
    %\item Новый способ порождения квазигрупп на основе правильных семейств функций; ряд утверждений о числе ассоциативных троек в порождаемых квазигруппах; новый алгоритм шифрования, сохраняющего формат ($\fpe$-схема), основанный на квазигрупповых операциях.
\end{enumerate}
% В папке Documents можно ознакомиться с решением совета из Томского~ГУ
% (в~файле \verb+Def_positions.pdf+), где обоснованно даются рекомендации
% по~формулировкам защищаемых положений.

{\reliability} полученных результатов обеспечивается строгими математическими доказательствами. 
Результаты работы докладывались на научных конференциях, опубликованы в рецензируемых научных журналах и находятся в соответствии с результатами, полученными другими авторами.
Результаты других авторов, используемые в диссертации, отмечены соответствующими ссылками.


{\probation}
Основные результаты работы докладывались~на следующих международных и всероссийских конференциях:
\begin{enumerate}
    \item XXVI Международная конференция студентов, аспирантов и молодых учёных <<Ломоносов>>, Москва, Россия, с 8 по 12 апреля 2019~г.;

    \item X симпозиум <<Современные тенденции в криптографии>> (CTCrypt 2021), Дорохово, Россия, с 1 по 4 июня 2021~г.;

    \item XI симпозиум <<Современные тенденции в криптографии>> (CTCrypt 2022), Новосибирск, Россия, с 6 по 9 июня 2022~г.;

    \item Четырнадцатый международный семинар <<Дискретная математика и ее приложения>> имени академика О.Б. Лупанова под руководством В.~В.~Кочергина, Э.~Э.~Гасанова, С.~А.~Ложкина, А.~В.~Чашкина, с 20 по 25 июня 2022~г.;

    \item 11-я Международная конференция <<Дискретные модели в теории управляющих систем>>, Красновидово, Россия, с 26 по 29 мая 2023~г.;

    \item Третья Международная конференция ``MATHEMATICS IN ARMENIA: ADVANCES AND PERSPECTIVES'', Ереван, Армения, со 2 по 8 июля 2023~г.;

    \item 22-я Международная конференция <<Сибирская научная школа-семинар ``Компьютерная безопасность и криптография'' имени Геннадия Петровича Агибалова>>, Барнаул, Россия, с 4 по 9 сентября 2023~г.;

    \item Международная конференция <<Математика в созвездии наук>>, Москва, Россия, с 1 по 2 апреля 2024~г.;

    \item Международная конференция <<Алгебра и математическая логика: теория и приложения>>, Казань, Россия, с 27 июня по 1 июля 2024~г.;

    \item XX Международная научная конференция <<Проблемы теоретической кибернетики>>, Москва, Россия, с 5 по 8 декабря 2024~г.
\end{enumerate}

Результаты работы докладывались и обсуждались на заседаниях следующих научных семинаров:
\begin{enumerate}    
    \item научно-исследовательский семинар по алгебре механико-математического факультета МГУ под руководством Д.~О.~Орлова, М.~В.~Зайцева, 2023~г.;

    \item научно-исследовательский семинар <<Математические вопросы кибернетики>> кафедр дискретной математики и математической теории интеллектуальных систем механико-математического факультета и математической кибернетики факультета вычислительной математики и кибернетики МГУ под руководством Э.~Э.~Гасанова, В.~В.~Кочергина, С.~А.~Ложкина, 2023~г.;

    \item семинар <<Компьютерная алгебра>> факультета ВМК МГУ и ВЦ РАН под руководством профессора С.~А.~Абрамова, 2023~г.;

    \item семинар <<Теория автоматов>> механико-математического факультета МГУ под руководством профессора Э.~Э.~Гасанова, 2023~г.;

    \item семинар <<Современные проблемы криптографии>> под руководством ведущего научного сотрудника В.~А.~Носова и доцента А.~Е.~Панкратьева, механико-математический факультет МГУ, неоднократно;

    \item семинар <<Компьютерная безопасность>> под руководством старшего научного сотрудника А.В. Галатенко, механико-математический факультет МГУ, неоднократно.
\end{enumerate}

% {\contribution} Автор принимал активное участие \ldots

\ifnumequal{\value{bibliosel}}{0}
{%%% Встроенная реализация с загрузкой файла через движок bibtex8. (При желании, внутри можно использовать обычные ссылки, наподобие `\cite{vakbib1,vakbib2}`).
    {\publications} Основные результаты по теме диссертации изложены
    в~XX~печатных изданиях,
    X из которых изданы в журналах, рекомендованных ВАК,
    X "--- в тезисах докладов.
}%
{%%% Реализация пакетом biblatex через движок biber
    \begin{refsection}[bl-author, bl-registered]
        % Это refsection=1.
        % Процитированные здесь работы:
        %  * подсчитываются, для автоматического составления фразы "Основные результаты ..."
        %  * попадают в авторскую библиографию, при usefootcite==0 и стиле `\insertbiblioauthor` или `\insertbiblioauthorgrouped`
        %  * нумеруются там в зависимости от порядка команд `\printbibliography` в этом разделе.
        %  * при использовании `\insertbiblioauthorgrouped`, порядок команд `\printbibliography` в нём должен быть тем же (см. biblio/biblatex.tex)
        %
        % Невидимый библиографический список для подсчёта количества публикаций:
        \printbibliography[heading=nobibheading, section=1, env=countauthorvak,          keyword=biblioauthorvak]%
        \printbibliography[heading=nobibheading, section=1, env=countauthorwos,          keyword=biblioauthorwos]%
        \printbibliography[heading=nobibheading, section=1, env=countauthorscopus,       keyword=biblioauthorscopus]%
        \printbibliography[heading=nobibheading, section=1, env=countauthorconf,         keyword=biblioauthorconf]%
        \printbibliography[heading=nobibheading, section=1, env=countauthorother,        keyword=biblioauthorother]%
        \printbibliography[heading=nobibheading, section=1, env=countregistered,         keyword=biblioregistered]%
        \printbibliography[heading=nobibheading, section=1, env=countauthorpatent,       keyword=biblioauthorpatent]%
        \printbibliography[heading=nobibheading, section=1, env=countauthorprogram,      keyword=biblioauthorprogram]%
        \printbibliography[heading=nobibheading, section=1, env=countauthor,             keyword=biblioauthor]%
        \printbibliography[heading=nobibheading, section=1, env=countauthorvakscopuswos, filter=vakscopuswos]%
        \printbibliography[heading=nobibheading, section=1, env=countauthorscopuswos,    filter=scopuswos]%
        %
        \nocite{*}%
        %
        {\publications} Основные результаты по теме диссертации изложены в~\arabic{citeauthor}~печатных изданиях,
        \arabic{citeauthorvak} из которых опубликованы в рецензируемых научных изданиях, рекомендованных для защиты в диссертационном совете МГУ по специальности 1.1.5. Математическая логика, алгебра, теория чисел и дискретная математика, из них 6 "--- в~ рецензируемых научных изданиях, входящих в ядро РИНЦ и международные базы цитирования (Web of Science / Scopus), RSCI, 2 "--- в~ рецензируемых научных изданиях из дополнительного списка МГУ, рекомендованных для защиты в диссертационном совете МГУ по специальности 1.1.5. Математическая логика, алгебра, теория чисел и дискретная математика и входящих в список ВАК.
        
        % %журналах, рекомендованных ВАК\sloppy%
        % \ifnum \value{citeauthorscopuswos}>0%
        %     , \arabic{citeauthorscopuswos} "--- в~периодических научных журналах, индексируемых Web of~Science и Scopus\sloppy%
        % \fi%
        % \ifnum \value{citeauthorconf}>0%
        %     , \arabic{citeauthorconf} "--- в~тезисах докладов.
        % \else%
        %     .
        % \fi%
        % \ifnum \value{citeregistered}=1%
        %     \ifnum \value{citeauthorpatent}=1%
        %         Зарегистрирован \arabic{citeauthorpatent} патент.
        %     \fi%
        %     \ifnum \value{citeauthorprogram}=1%
        %         Зарегистрирована \arabic{citeauthorprogram} программа для ЭВМ.
        %     \fi%
        % \fi%
        % \ifnum \value{citeregistered}>1%
        %     Зарегистрированы\ %
        %     \ifnum \value{citeauthorpatent}>0%
        %     \formbytotal{citeauthorpatent}{патент}{}{а}{}\sloppy%
        %     \ifnum \value{citeauthorprogram}=0 . \else \ и~\fi%
        %     \fi%
        %     \ifnum \value{citeauthorprogram}>0%
        %     \formbytotal{citeauthorprogram}{программ}{а}{ы}{} для ЭВМ.
        %     \fi%
        % \fi%
        % К публикациям, в которых излагаются основные научные результаты диссертации на соискание учёной
        % степени, в рецензируемых изданиях приравниваются патенты на изобретения, патенты (свидетельства) на
        % полезную модель, патенты на промышленный образец, патенты на селекционные достижения, свидетельства
        % на программу для электронных вычислительных машин, базу данных, топологию интегральных микросхем,
        % зарегистрированные в установленном порядке.(в ред. Постановления Правительства РФ от 21.04.2016 N 335)
    \end{refsection}%
    \begin{refsection}[bl-author, bl-registered]
        % Это refsection=2.
        % Процитированные здесь работы:
        %  * попадают в авторскую библиографию, при usefootcite==0 и стиле `\insertbiblioauthorimportant`.
        %  * ни на что не влияют в противном случае
        \nocite{intsys20}
        \nocite{pdm20}
        \nocite{dm21}
        \nocite{fpe22}
        \nocite{galatenko23}
        \nocite{galatenko2023proper}
        \nocite{fpm23}
        \nocite{sibecrypt23}
        \nocite{tsar24}
        %conference papers
        % \nocite{lomonosov19}
        % \nocite{ctcrypt21}
        % \nocite{ctcrypt22}
        % \nocite{dmapp22}
        % \nocite{krasnovidovo23}
        % \nocite{armenia23}
        % \nocite{sibecrypt23}
        % \nocite{sozvezdie24}
        % \nocite{sozvezdie24_2}
        % \nocite{msu24}
    \end{refsection}%
        %
        % Всё, что вне этих двух refsection, это refsection=0,
        %  * для диссертации - это нормальные ссылки, попадающие в обычную библиографию
        %  * для автореферата:
        %     * при usefootcite==0, ссылка корректно сработает только для источника из `external.bib`. Для своих работ --- напечатает "[0]" (и даже Warning не вылезет).
        %     * при usefootcite==1, ссылка сработает нормально. В авторской библиографии будут только процитированные в refsection=0 работы.
}
