%%% Основные сведения %%%
\newcommand{\thesisAuthorLastName}{Царегородцев}
\newcommand{\thesisAuthorOtherNames}{Кирилл Денисович}
\newcommand{\thesisAuthorInitials}{К.\,Д.}
\newcommand{\thesisAuthor}             % Диссертация, ФИО автора
{%
    \texorpdfstring{% \texorpdfstring takes two arguments and uses the first for (La)TeX and the second for pdf
        \thesisAuthorLastName~\thesisAuthorOtherNames% так будет отображаться на титульном листе или в тексте, где будет использоваться переменная
    }{%
        \thesisAuthorLastName, \thesisAuthorOtherNames% эта запись для свойств pdf-файла. В таком виде, если pdf будет обработан программами для сбора библиографических сведений, будет правильно представлена фамилия.
    }
}
\newcommand{\thesisAuthorShort}        % Диссертация, ФИО автора инициалами
{\thesisAuthorInitials~\thesisAuthorLastName}
%\newcommand{\thesisUdk}{01.01.06}                % Диссертация, УДК
%{\}
\newcommand{\thesisTitle}              % Диссертация, название
{Правильные семейства функций и порождаемые ими квазигруппы: комбинаторные и алгебраические свойства}
\newcommand{\thesisSpecialtyNumber}    % Диссертация, специальность, номер
{1.1.5}
\newcommand{\thesisSpecialtyTitle}     % Диссертация, специальность, название (название взято с сайта ВАК для примера)
{Математическая логика, алгебра, теория чисел и дискретная математика}
%% \newcommand{\thesisSpecialtyTwoNumber} % Диссертация, вторая специальность, номер
%% {\fixme{XX.XX.XX}}
%% \newcommand{\thesisSpecialtyTwoTitle}  % Диссертация, вторая специальность, название
%% {\fixme{Теория и~методика физического воспитания, спортивной тренировки,
%% оздоровительной и~адаптивной физической культуры}}
\newcommand{\thesisDegree}             % Диссертация, ученая степень
{кандидата физико-математических наук}
\newcommand{\thesisDegreeShort}        % Диссертация, ученая степень, краткая запись
{канд. физ.-мат. наук}
\newcommand{\thesisCity}               % Диссертация, город написания диссертации
{Москва}
\newcommand{\thesisYear}               % Диссертация, год написания диссертации
{\the\year}
\newcommand{\thesisOrganization}       % Диссертация, организация
{МОСКОВСКИЙ ГОСУДАРСТВЕННЫЙ УНИВЕРСИТЕТ \par имени М.В. Ломоносова}
\newcommand{\thesisOrganizationShort}  % Диссертация, краткое название организации для доклада
{МГУ}

\newcommand{\thesisInOrganization}     % Диссертация, организация в предложном падеже: Работа выполнена в ...
{кафедре математической теории интеллектуальных систем механико-математического факультета МГУ имени М.В. Ломоносова}

%% \newcommand{\supervisorDead}{}           % Рисовать рамку вокруг фамилии
\newcommand{\supervisorFio}              % Научный руководитель, ФИО
{Панкратьев Антон Евгеньевич}
\newcommand{\supervisorRegalia}          % Научный руководитель, регалии
{кандидат физико-математических наук}
\newcommand{\supervisorFioShort}         % Научный руководитель, ФИО
{А.\,Е.~Панкратьев}
\newcommand{\supervisorRegaliaShort}     % Научный руководитель, регалии
{к.~ф.-м.~н.}

%% \newcommand{\supervisorTwoDead}{}        % Рисовать рамку вокруг фамилии
\newcommand{\supervisorTwoFio}           % Второй научный руководитель, ФИО
{Галатенко Алексей Владимирович}
\newcommand{\supervisorTwoRegalia}       % Второй научный руководитель, регалии
{кандидат физико-математических наук}
\newcommand{\supervisorTwoFioShort}      % Второй научный руководитель, ФИО
{А.\,В.~Галатенко}
\newcommand{\supervisorTwoRegaliaShort}  % Второй научный руководитель, регалии
{к.~ф.-м.~н.}

\newcommand{\opponentOneFio}           % Оппонент 1, ФИО
{Щучкин Николай Алексеевич}
\newcommand{\opponentOneRegalia}       % Оппонент 1, регалии
{доктор физико-математических наук, доцент}
\newcommand{\opponentOneJobPlace}      % Оппонент 1, место работы
{Волгоградский государственный социально-педагогический университет, кафедра высшей математики и физики Института математики, информатики и физики}
\newcommand{\opponentOneJobPost}       % Оппонент 1, должность
{профессор}

\newcommand{\opponentTwoFio}           % Оппонент 2, ФИО
{Камловский Олег Витальевич}
\newcommand{\opponentTwoRegalia}       % Оппонент 2, регалии
{доктор физико-математических наук, доцент}
\newcommand{\opponentTwoJobPlace}      % Оппонент 2, место работы
{Московский технический университет связи и информатики, кафедра теории вероятностей и прикладной математики}
\newcommand{\opponentTwoJobPost}       % Оппонент 2, должность
{профессор}

\newcommand{\opponentThreeFio}         % Оппонент 3, ФИО
{Токарева Наталья Николаевна}
\newcommand{\opponentThreeRegalia}     % Оппонент 3, регалии
{кандидат физико-математических наук, доцент}
\newcommand{\opponentThreeJobPlace}    % Оппонент 3, место работы
{Новосибирский государственный университет, кафедра теоретической кибернетики механико-математического факультета}
\newcommand{\opponentThreeJobPost}     % Оппонент 3, должность
{ведущий научный сотрудник}

%\newcommand{\leadingOrganizationTitle} % Ведущая организация, дополнительные строки. Удалить, чтобы не отображать в автореферате
%{\fixme{Федеральное государственное бюджетное образовательное учреждение высшего
%профессионального образования с~длинным длинным длинным длинным названием}}

\newcommand{\defenseDate}              % Защита, дата
{\fixme{DD mmmmmmmm YYYY~г.~в~XX часов}}
\newcommand{\defenseCouncilNumber}     % Защита, номер диссертационного совета
{МГУ.011.4-1}
\newcommand{\defenseCouncilTitle}      % Защита, учреждение диссертационного совета
{\mbox{ФГБОУ ВО} <<Московский государственный университет имени М. В. Ломоносова>>}
\newcommand{\defenseCouncilAddress}    % Защита, адрес учреждение диссертационного совета
{Российская Федерация, 119234, Москва, ГСП-1, Ленинские горы, д. 1, МГУ имени М. В. Ломоносова, механико-математический факультет, аудитория~\fixme{XX}}
\newcommand{\defenseCouncilPhone}      % Телефон для справок
{dissovet.msu.011.4@math.msu.ru}

\newcommand{\defenseSecretaryFio}      % Секретарь диссертационного совета, ФИО
{В.А. Кибкало}
\newcommand{\defenseSecretaryRegalia}  % Секретарь диссертационного совета, регалии
{кандидат~физико-математических наук}            % Для сокращений есть ГОСТы, например: ГОСТ Р 7.0.12-2011 + http://base.garant.ru/179724/#block_30000

\newcommand{\synopsisLibrary}          % Автореферат, название библиотеки
{отделе диссертаций научной библиотеки МГУ имени М. В. Ломоносова (Ломоносовский просп., д. 27)}
\newcommand{\synopsisDate}             % Автореферат, дата рассылки
{\fixme{DD mmmmmmmm}~\the\year~года}

% To avoid conflict with beamer class use \providecommand
\providecommand{\keywords}%            % Ключевые слова для метаданных PDF диссертации и автореферата
{}
