%% Согласно ГОСТ Р 7.0.11-2011:
%% 5.3.3 В заключении диссертации излагают итоги выполненного исследования, рекомендации, перспективы дальнейшей разработки темы.
%% 9.2.3 В заключении автореферата диссертации излагают итоги данного исследования, рекомендации и перспективы дальнейшей разработки темы.
\begin{enumerate}
  \item Доказано, что правильные семейства булевых функций находятся во взаимно-однозначном соответствии с одностоковыми ориентациями графов булевых кубов.
  \item Доказано, что правильные семейства булевых функций находятся во взаимно-однозначном соответствии с булевыми сетями с наследственно единственной неподвижной точкой.
  \item Доказано, что стабилизатором множества правильных семейств булевых функций являются изометрии пространства Хэмминга (согласованные перенумерации и перекодировки).
  \item Показано, что отображения, задаваемые с помощью правильных семейств булевых функций, всегда имеют четное число неподвижных точек.
  \item Получена нижняя оценка на число правильных семейств булевых функций.
  \item Предложены оценки доли треугольных семейств среди всех правильных семейств булевых функций.
  \item Обнаружены новые классы правильных семейств булевых функций, доказаны некоторые их свойства.
  \item Предложен новый алгоритм шифрования, сохраняющего формат, основанный на квазигрупповых операциях.
\end{enumerate}

В качестве тем для дальнейших исследований можно отметить следующие направления.

\begin{enumerate}
  \item Предложить способ построения достаточно широких классов правильных семейств с хорошими алгебраическими и комбинаторными свойствами, в том числе и для логик большей значности $k > 2$.

  \item Предложить способ быстрого построения множества представителей всех правильных семейств размера $n+1$ с помощью представителей размера $n$ и менее (с точностью до согласованных перенумераций и перекодировок).

  \item Предложить альтернативные геометрические описания правильных семейств в $k$-значной логике, где $k>2$, которые были бы инвариантны относительно согласованных перенумераций и перекодировок.

  \item Предложить алгоритм, полиномиальный по длине входа, на вход принимающий правильное семейство (например, в виде КНФ или полиномов Жегалкина) и параметрические подстановки и выдающий количество ассоциативных троек (или нижние и верхние границы на число троек), проверяющий полиномиальную полноту порождаемой квазигруппы, наличие или отсутствие подквазигрупп.

  \item Оценить генерическую сложность задачи решения системы уравнений над квазигруппами, заданными правильными семействами.
\end{enumerate}
