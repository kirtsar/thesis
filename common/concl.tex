%% Согласно ГОСТ Р 7.0.11-2011:
%% 5.3.3 В заключении диссертации излагают итоги выполненного исследования, рекомендации, перспективы дальнейшей разработки темы.
%% 9.2.3 В заключении автореферата диссертации излагают итоги данного исследования, рекомендации и перспективы дальнейшей разработки темы.

    \begin{enumerate}
        \item Установлено естественное соответствие между булевыми правильными семействами и одностоковыми ориентациями графов булевых кубов ($\uso$-ориентации).
        \item Установлено естественное соответствие между булевыми правильными семействами и булевыми сетями с наследственно единственной неподвижной точкой ($\hupf$-сети).
        \item Установлено естественное соответствие между правильными семействами в логике произвольной значности и кликами в обобщенных графах Келлера.
        \item Доказано, что стабилизатором множества правильных семейств функций являются изометрии пространства Хэмминга (согласованные перенумерации и перекодировки).
        \item Показано, что отображения, задаваемые с помощью правильных семейств булевых функций, всегда имеют четное число неподвижных точек.
        \item Получена оценка на число правильных семейств булевых функций, предложены оценки доли треугольных семейств среди всех правильных семейств булевых функций.
        \item Обнаружены и исследованы новые классы правильных семейств функций (рекурсивно треугольные, локально треугольные, сильно квадратичное семейство).
        \item Получены оценки на число рекурсивно треугольных семейств.
        \item Для некоторых правильных семейств булевых функций получены точные значения мощности образа отображений, задаваемых этими правильными семействами.
        \item Предложен новый способ порождения квазигрупп на основе правильных семейств функций.
        \item Доказан ряд утверждений о числе ассоциативных троек в порождаемых квазигруппах.
        \item Предложен новый алгоритм шифрования, сохраняющего формат ($\fpe$-схема), основанный на квазигрупповых операциях.
    \end{enumerate}

    Результаты диссертационной работы могут представлять интерес для специалистов, работающих в области теории дискретных и булевых функций, теории квазигрупп, криптографии.

    В качестве тем для дальнейших исследований можно отметить следующие направления.

    \begin{enumerate}
        \item Предложить способ построения достаточно широких классов правильных семейств с хорошими алгебраическими и комбинаторными свойствами, в том числе и для логик большей значности $k > 2$.

        \item Предложить способ быстрого построения множества представителей всех правильных семейств размера $n+1$ с помощью представителей размера $n$ и менее (с точностью до согласованных перенумераций и перекодировок).

        \item Предложить альтернативные геометрические описания правильных семейств в $k$-значной логике, где $k>2$, которые были бы инвариантны относительно согласованных перенумераций и перекодировок.

        \item Предложить алгоритм, полиномиальный по длине входа, на вход принимающий правильное семейство (например, в виде КНФ или полиномов Жегалкина) и параметрические подстановки и выдающий количество ассоциативных троек (или нижние и верхние границы на число троек), проверяющий полиномиальную полноту порождаемой квазигруппы, наличие или отсутствие подквазигрупп.

        \item Оценить генерическую сложность задачи решения системы уравнений над квазигруппами, заданными правильными семействами.
    \end{enumerate}
