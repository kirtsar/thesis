\pdfbookmark{Общая характеристика работы}{characteristic}             % Закладка pdf
\section*{Общая характеристика работы}

\newcommand{\actuality}{\pdfbookmark[1]{Актуальность}{actuality}\underline{\textbf{\actualityTXT}}}
\newcommand{\progress}{\pdfbookmark[1]{Разработанность темы}{progress}\underline{\textbf{\progressTXT}}}
\newcommand{\aim}{\pdfbookmark[1]{Цели}{aim}\underline{{\textbf\aimTXT}}}
\newcommand{\tasks}{\pdfbookmark[1]{Задачи}{tasks}\underline{\textbf{\tasksTXT}}}
\newcommand{\aimtasks}{\pdfbookmark[1]{Цели и задачи}{aimtasks}\aimtasksTXT}
\newcommand{\novelty}{\pdfbookmark[1]{Научная новизна}{novelty}\underline{\textbf{\noveltyTXT}}}
\newcommand{\influence}{\pdfbookmark[1]{Практическая значимость}{influence}\underline{\textbf{\influenceTXT}}}
\newcommand{\methods}{\pdfbookmark[1]{Методология и методы исследования}{methods}\underline{\textbf{\methodsTXT}}}
\newcommand{\defpositions}{\pdfbookmark[1]{Положения, выносимые на защиту}{defpositions}\underline{\textbf{\defpositionsTXT}}}
\newcommand{\reliability}{\pdfbookmark[1]{Достоверность}{reliability}\underline{\textbf{\reliabilityTXT}}}
\newcommand{\probation}{\pdfbookmark[1]{Апробация}{probation}\underline{\textbf{\probationTXT}}}
\newcommand{\contribution}{\pdfbookmark[1]{Личный вклад}{contribution}\underline{\textbf{\contributionTXT}}}
\newcommand{\publications}{\pdfbookmark[1]{Публикации}{publications}\underline{\textbf{\publicationsTXT}}}

%\newcommand{\thm}[1]{\textbf{Теорема #1}}


% актуальность
{\actuality} 

    Квазигруппы~--- одни из базовых структур в алгебре.
    Таблицы умножения квазигрупп, более известные под названием \textquote{латинские квадраты}, с древнейших времен и по настоящее время используются в различных областях математики~\autocite{keedwell}: при планировании статистических экспериментов, в играх и головоломках, а также в теории кодирования и криптографии, которые рассматриваются более подробно в настоящей работе.
    Из общих обзоров криптографических приложений квазигрупп можно отметить следующие источники:
    \begin{itemize}
        \item статья~\autocite{glukhov}, в которой приводятся примеры кодов аутентификации, шифров и однонаправленных функций на основе квазигрупповых преобразований, а также недавний обзор~\autocite{chauhan2021quasigroups}, затрагивающий тематику построения симметричных криптопримитивов на основе квазигрупповых операций;
        \item монография~\autocite{shcherbacov2017elements}, в которой довольно подробно освещена тематика использования квазигрупп в криптографии; в частности, в работе рассматриваются следующие темы: поточные шифры и их криптоанализ, хэш-функции и односторонние функции, схемы разделения секрета; а также смежная тематика теории кодирования (в частности, рекурсивные МДР-коды);
        \item монография~\autocite{keedwell} и статья~\autocite{tuzhilin2012}, посвященные общим обзорам тематики латинских квадратов, их использованию в докомпьютерный этап развития криптографии и современным приложениям.
    \end{itemize}

    В качестве непосредственного приложения квазигрупп в области симметричной криптографии можно привести следующие механизмы, основанные на квазигрупповых операциях:
    \begin{itemize}
        \item поточные шифры и хэш-функции, основанные на квазигрупповом умножении~\autocite{markovski1999quasigroup, markovski2003quasigroup, markovski2017quasigroup, snavsel2009hash},
        \item кандидат на стандартизацию в качестве поточного шифра \textbf{Edon80}~\autocite{edon80},
        \item кандидаты на стандартизацию в качестве хэш-функции \textbf{Edon-R}~\autocite{EdonR, EdonRprime} и \textbf{NaSHA}~\autocite{nasha, mileva2009quasigroup},
        \item кандидаты на стандартизацию в качестве низкоресурсной хэш-функции и алгоритма шифрования с ассоциированными (присоединенными) данными (AEAD-алгоритм) \textbf{GAGE} и \textbf{InGAGE}~\autocite{otte2019gage, gligoroski2019s},
        \item предложения по использованию квазигрупповых операций в рамках (обобщенных) сетей Фейстеля~\autocite{tecseleanu2021quasigroups, tecseleanu2022security, tecseleanu2023cryptographic, cherednik17, cherednik19, cherednik20}.
    \end{itemize}
    Однако недостаточная изученность задач, лежащих в основании подобных предложений, иногда приводит к возможности довольно простого криптоанализа полученных решений~\autocite{vojvoda2004cryptanalysis, slaminkova2010cryptanalysis, hell2007key, vojvoda2007note, nikolicfree, li2010collision}.
    %В качестве примеров применения квазигрупп в практических приложениях можно отметить следующие предложения:    

    Квазигруппы (а также более сложные алгебраические структуры, в основе которых лежат квазигруппы) используются и в теории кодирования.
    Так, в статье \autocite{nechaev98} исследуются $k$-рекурсивные коды (т.е. коды, для которых позиции в кодовых словах с номерами $i+k$ однозначно определяются по позициям $i, i+1, \ldots, i+k-1$ для $i = k+1, \ldots, n-k,$ другими словами $u_{i+k} = f(u_i, \ldots, u_{i+k-1})$), лежащие на границе Синглтона (МДР-коды).
    Подход, основанный на применении ортогональных латинских квадратов, позволяет получить в данном случае оценки на максимальную длину кодовых слов.
    В серии работ~\autocite{nechaev04, couselo2004loop, markov12, markov2020nonassociative} используются так называемые луповые кольца (формальные суммы квазигрупповых элементов) для построения различных оптимальных в разных смыслах кодов.

    Луповые кольца и другие алгебраические структуры, основанные на квазигруппах, могут быть использованы для построения множества асимметричных криптографических примитивов, среди которых можно выделить следующие:
    \begin{itemize}
        \item протоколы формирования общего ключа (аналоги протокола Диффи-Хеллмана)~\autocite{DH14, DH16, baryshnikov2017key, quantum18},
        \item схемы асимметричного шифрования~\autocite{markov12, pke10, gribovphd},
        \item схемы гомоморфного шифрования~\autocite{gribovphd, homomo15, katyshev2020application, markov20},
        \item схемы асимметричного шифрования и цифровой подписи, основанные на сложности решений систем уравнений в конечных полях~\autocite{gligoroski2008public, gligoroski2008multivariate, chen2010multivariate, gligoroski2011mqq}.
    \end{itemize}

    При этом применяемые в области защиты информации квазигруппы часто имеют довольно большие размеры (см., например, требования к квазигруппе в работах~\autocite{EdonR, EdonRprime, chen2010multivariate}), что делает затруднительным поэлементное хранение в памяти компьютера всей таблицы умножения. 
    Так, например, в~\autocite{EdonRprime} для построения хэш-функции необходимо задать квазигруппу порядка $2^{256}$. 
    В связи с этим обстоятельством в большинстве предлагаемых криптосистем большая квазигруппа строится, как правило, согласно одному из следующих подходов:
    \begin{itemize}
        \item случайная генерация квазигрупп (случайных поиск подходящей квазигруппы совместно с процедурой отсева неподходящих) из некоторого узкого класса~\autocite{gligoroski2008public, chen2010multivariate},
        \item итеративное построение большой квазигруппы из квазигрупп меньшего размера~\autocite{EdonRprime, gribovphd} с помощью конструкций произведений,
        \item изотопы некоторых \textquote{хорошо изученных} групп (например, изотоп группы точек эллиптической кривой~\autocite{DH16}, модульное вычитание~\autocite{snavsel2009hash}),
        \item функциональное задание квазигруппы, рассматриваемое в настоящей работе более подробно.
    \end{itemize}

    В работах~\autocite{nosov98, nosov99} был предложен метод задания латинского квадрата при помощи семейства булевых функций, которое определяет элемент квадрата (произведение) по его координатам (номеру строки и столбца).
    Такие семейства функций, задающие целые параметрические классы латинских квадратов, были названы правильными.
    Понятие правильного семейства функций было сначала обобщено на случай абелевых групп~\autocite{nosov06, nosov06abel, nosov07, nosov08, kozlov08}, а затем и на более общие алгебраические структуры~\autocite{plaksina14, galatenko2020latin, galatenko23}.
    Ряд работ посвящен изучению свойств введенных булевых отображений:
    \begin{itemize}
        \item в~\autocite{nosov98} было (среди прочего) показано, что проверка свойства правильности является $\mathsf{coNP}$-полной задачей (т.е. в общем случае задача проверки правильности является сложной),
        \item в~\autocite{nosov07, nosov08, kozlov08} рассматривались свойства т.н. графа существенной зависимости правильных семейств (граф на $n$ вершинах, ребро $ i \to j$ присутствует в графе тогда и только тогда, когда $j$-я функция семейства зависит существенно от $x_i$) и были выделены широкие классы семейств, для которых свойство правильности эквивалентно свойству отсутствия циклов в графе существенной зависимости,
        \item в~\autocite{rykov10, rykov14} показано, как задача проверки свойства правильности может быть упрощена, если дополнительно известна структура графа существенной зависимости семейства,
        \item работы~\autocite{plaksina14, galatenko2020latin, galatenko23} посвящены, в том числе, различным способам задания ($d$-)квазигрупп с помощью правильных семейств над различными алгебраическими структурами,
        \item работы~\autocite{galatenko21generation, galatenko2022generation} посвящены вопросам построения новых правильных семейств функций из старых.
    \end{itemize}

    При этом не всякая квазигруппа подходит для реализации на ее основе криптографических примитивов.
    Критически важными являются алгебраические свойства используемой квазигруппы, такие как свойства полиномиальной полноты~\autocite{hagemann, nipkow1990unification, horvath2008}, количество ассоциативных троек~\autocite{kepka1980note, kotzig83, ass_summary}, наличие подквазигрупп~\autocite{sobyanin19, galatenko21subquasi}.
    В работах~\autocite{piven18, piven19, galatenko20quad, shvaryov24} изучаются свойства квазигрупп, порождаемых правильными семействами булевых функций:
    \begin{itemize}
        \item в~\autocite{piven18} исследуются алгебраические свойства квазигрупп размера~4, порождаемых правильными семействами булевых функций размера $n = 2$, вводится понятие \textquote{перестановочной конструкции} (способ получения новых квазигрупп из уже имеющихся),
        \item в~\autocite{piven19} рассмотрена избыточность \textquote{перестановочной конструкции} (различные значения параметров могут давать одну и ту же квазигруппу) и способы сокращения избыточности,
        \item в~\autocite{galatenko20quad} предложен способ построения квадратичных квазигрупп, которые являются оптимальными с точки зрения криптографических приложений (обладают наиболее компактным представлением, при этом задача решения систем уравнений над подобными квазигруппами является в общем случае сложной),
        \item в работе~\autocite{shvaryov24}, среди прочего, рассмотрены \textquote{криптографические} свойств квазигрупп, порождаемых правильными семействами (линейная, дифференциальная характеристики) и способы их \textquote{усиления}.
    \end{itemize}

    Таким образом, возникает ряд нерешенных задач, которым посвящена настоящая работа:
    \begin{itemize}
        \item изучение правильных семейств и их свойств как одного из возможных способов функционального задания квазигрупповой операции,
        \item изучение свойств квазигрупп, порождаемых правильными семействами.
    \end{itemize}

\ifsynopsis
%Этот абзац появляется только в~автореферате.
\else
% Этот абзац появляется только в~диссертации.
\fi

% % цель работы
{\aim} исследования является изучение свойств правильных семейств функций, а также алгебраических свойств квазигрупп, заданных правильными семействами булевых функций.
Тема, объект и предмет диссертационной работы соответствуют следующим пунктам паспорта специальности 1.1.5~--- Математическая логика, алгебра, теория чисел и дискретная математика: теория алгебраических структур (полугрупп, групп, колец, полей, модулей и т.д.), теория дискретных функций и автоматов, теория графов и комбинаторика. 
\TODO{теория сложности вычислений сюда плохо ложится, она относится к теоретической информатике.}


Для~достижения поставленной цели необходимо было решить следующие {\tasks}:
\begin{enumerate}[beginpenalty=10000] % https://tex.stackexchange.com/a/476052/104425
  \item \TODO{непонятно, как писать...}
\end{enumerate}


% научная новизна
{\novelty}
Результаты диссертации являются новыми и получены автором самостоятельно. 
Основные результаты диссертации состоят в следующем.
\begin{enumerate}[beginpenalty=10000] % https://tex.stackexchange.com/a/476052/104425
  \item Доказано, что правильные семейства булевых функций находятся во взаимно-однозначном соответствии с одностоковыми ориентациями графов булевых кубов и с с булевыми сетями с наследственно единственной неподвижной точкой.
  \item Доказано, что правильные семейства функций находятся во взаимно-однозначном соответствии с кликами в обобщенных графах Келлера.
  \item Доказано, что стабилизатором множества правильных семейств булевых функций являются изометрии пространства Хэмминга (согласованные перенумерации и перекодировки).
  \item Показано, что отображения, задаваемые с помощью правильных семейств булевых функций, всегда имеют четное число неподвижных точек.
  \item Получена нижняя оценка на число правильных семейств булевых функций, предложены оценки доли треугольных семейств среди всех правильных семейств булевых функций.
  \item Обнаружены и исследованы новые классы правильных семейств функций (рекурсивно треугольные, локально треугольные, сильно квадратичное семейство).
  \item Предложен новый алгоритм шифрования, сохраняющего формат, основанный на квазигрупповых операциях.
\end{enumerate}

%{\influence} \ldots

{\methods} В работе используются методы алгебры, дискретной математики, криптографии, теории графов, теории сложности.

{\defpositions}
\begin{enumerate}[beginpenalty=10000] % https://tex.stackexchange.com/a/476052/104425
  \item Первое положение
  \item Второе положение
  \item Третье положение
  \item Четвертое положение
\end{enumerate}
% В папке Documents можно ознакомиться с решением совета из Томского~ГУ
% (в~файле \verb+Def_positions.pdf+), где обоснованно даются рекомендации
% по~формулировкам защищаемых положений.

{\reliability} полученных результатов обеспечивается строгими математическими доказательствами. 
Результаты работы докладывались на научных конференциях, опубликованы в рецензируемых научных журналах и находятся в соответствии с результатами, полученными другими авторами.


{\probation}
Основные результаты работы докладывались~на следующих конференциях:
\begin{enumerate}
    \item XXVI Международная конференция студентов, аспирантов и молодых учёных <<Ломоносов>>, Москва, Россия, 2019~г.;

    \item X симпозиум <<Современные тенденции в криптографии>> (CTCrypt 2021), Дорохово, Россия, 2021~г.;

    \item XI симпозиум <<Современные тенденции в криптографии>> (CTCrypt 2022), Новосибирск, Россия, 2022~г.;

    \item 14-й Международный научный семинар <<Дискретная математика и ее приложения>> им. акад. О.Б.Лупанова, Москва, Россия, 2022~г.;

    \item 11-я Международная конференция <<Дискретные модели в теории управляющих систем>>, Красновидово, Россия, 2023~г;

    \item Третья Международная конференция ``MATHEMATICS IN ARMENIA: ADVANCES AND PERSPECTIVES'', Ереван, Армения, 2023~г;

    \item 22-я Международная конференция <<Сибирская научная школа-семинар ``Компьютерная безопасность и криптография'' имени Геннадия Петровича Агибалова>>, Барнаул, Россия, 2023~г.;

    \item Международная конференция <<Математика в созвездии наук>>, Москва, Россия, 2024~г.;

    \item Международная конференция <<Алгебра и математическая логика: теория и приложения>>, Казань, Россия, 2024~г.;

    \item XX Международная научная конференция <<Проблемы теоретической кибернетики>>, Москва, Россия, 2024~г.
\end{enumerate}

Результаты работы докладывались и обсуждались на заседаниях следующих научных семинаров:
\begin{enumerate}
    \item научно-исследовательский семинар по алгебре механико-математического факультета МГУ под руководством \TODO{вариант с math-net: Артамонова В.~А., Буниной Е.~И., Винберга Э.~Б., Голода Е.~С., Гутермана А.~Э., Зайцева М.~В., Латышева В.~Н., Михалёва А.~В.}, 2023~г.;
    \TODO{мой вариант: Орлова Д.~О., Гутермана А.~Э. (?), Зайцева М.~В.}

    \item семинар <<Компьютерная алгебра>> факультета ВМК МГУ и ВЦ РАН под руководством профессора С.А. Абрамова, 2023~г.;

    \item семинар <<Теория автоматов>> механико-математического факультета МГУ под руководством профессора Э.Э. Гасанова, 2023~г.;

    \item семинар <<Современные проблемы криптографии>> под руководством ведущего научного сотрудника В.А. Носова и доцента А.Е. Панкратьева, механико-математический факультет МГУ имени М.В. Ломоносова, неоднократно;

    \item семинар <<Компьютерная безопасность>> под руководством старшего научного сотрудника А.В. Галатенко, механико-математический факультет МГУ имени М.В. Ломоносова, неоднократно;
\end{enumerate}

% {\contribution} Автор принимал активное участие \ldots

\ifnumequal{\value{bibliosel}}{0}
{%%% Встроенная реализация с загрузкой файла через движок bibtex8. (При желании, внутри можно использовать обычные ссылки, наподобие `\cite{vakbib1,vakbib2}`).
    {\publications} Основные результаты по теме диссертации изложены
    в~XX~печатных изданиях,
    X из которых изданы в журналах, рекомендованных ВАК,
    X "--- в тезисах докладов.
}%
{%%% Реализация пакетом biblatex через движок biber
    \begin{refsection}[bl-author, bl-registered]
        % Это refsection=1.
        % Процитированные здесь работы:
        %  * подсчитываются, для автоматического составления фразы "Основные результаты ..."
        %  * попадают в авторскую библиографию, при usefootcite==0 и стиле `\insertbiblioauthor` или `\insertbiblioauthorgrouped`
        %  * нумеруются там в зависимости от порядка команд `\printbibliography` в этом разделе.
        %  * при использовании `\insertbiblioauthorgrouped`, порядок команд `\printbibliography` в нём должен быть тем же (см. biblio/biblatex.tex)
        %
        % Невидимый библиографический список для подсчёта количества публикаций:
        \printbibliography[heading=nobibheading, section=1, env=countauthorvak,          keyword=biblioauthorvak]%
        \printbibliography[heading=nobibheading, section=1, env=countauthorwos,          keyword=biblioauthorwos]%
        \printbibliography[heading=nobibheading, section=1, env=countauthorscopus,       keyword=biblioauthorscopus]%
        \printbibliography[heading=nobibheading, section=1, env=countauthorconf,         keyword=biblioauthorconf]%
        \printbibliography[heading=nobibheading, section=1, env=countauthorother,        keyword=biblioauthorother]%
        \printbibliography[heading=nobibheading, section=1, env=countregistered,         keyword=biblioregistered]%
        \printbibliography[heading=nobibheading, section=1, env=countauthorpatent,       keyword=biblioauthorpatent]%
        \printbibliography[heading=nobibheading, section=1, env=countauthorprogram,      keyword=biblioauthorprogram]%
        \printbibliography[heading=nobibheading, section=1, env=countauthor,             keyword=biblioauthor]%
        \printbibliography[heading=nobibheading, section=1, env=countauthorvakscopuswos, filter=vakscopuswos]%
        \printbibliography[heading=nobibheading, section=1, env=countauthorscopuswos,    filter=scopuswos]%
        %
        \nocite{*}%
        %
        {\publications} Основные результаты по теме диссертации изложены в~\arabic{citeauthor}~печатных изданиях,
        \arabic{citeauthorvak} из которых изданы в журналах, рекомендованных ВАК\sloppy%
        \ifnum \value{citeauthorscopuswos}>0%
            , \arabic{citeauthorscopuswos} "--- в~периодических научных журналах, индексируемых Web of~Science и Scopus\sloppy%
        \fi%
        \ifnum \value{citeauthorconf}>0%
            , \arabic{citeauthorconf} "--- в~тезисах докладов.
        \else%
            .
        \fi%
        \ifnum \value{citeregistered}=1%
            \ifnum \value{citeauthorpatent}=1%
                Зарегистрирован \arabic{citeauthorpatent} патент.
            \fi%
            \ifnum \value{citeauthorprogram}=1%
                Зарегистрирована \arabic{citeauthorprogram} программа для ЭВМ.
            \fi%
        \fi%
        \ifnum \value{citeregistered}>1%
            Зарегистрированы\ %
            \ifnum \value{citeauthorpatent}>0%
            \formbytotal{citeauthorpatent}{патент}{}{а}{}\sloppy%
            \ifnum \value{citeauthorprogram}=0 . \else \ и~\fi%
            \fi%
            \ifnum \value{citeauthorprogram}>0%
            \formbytotal{citeauthorprogram}{программ}{а}{ы}{} для ЭВМ.
            \fi%
        \fi%
        % К публикациям, в которых излагаются основные научные результаты диссертации на соискание учёной
        % степени, в рецензируемых изданиях приравниваются патенты на изобретения, патенты (свидетельства) на
        % полезную модель, патенты на промышленный образец, патенты на селекционные достижения, свидетельства
        % на программу для электронных вычислительных машин, базу данных, топологию интегральных микросхем,
        % зарегистрированные в установленном порядке.(в ред. Постановления Правительства РФ от 21.04.2016 N 335)
    \end{refsection}%
    \begin{refsection}[bl-author, bl-registered]
        % Это refsection=2.
        % Процитированные здесь работы:
        %  * попадают в авторскую библиографию, при usefootcite==0 и стиле `\insertbiblioauthorimportant`.
        %  * ни на что не влияют в противном случае
        \nocite{intsys20}
        \nocite{pdm20}
        \nocite{dm21}
        \nocite{fpe22}
        \nocite{galatenko23}
        \nocite{galatenko2023proper}
        \nocite{fpm23}
        \nocite{tsar24}
        %conference papers
        % \nocite{lomonosov19}
        % \nocite{ctcrypt21}
        % \nocite{ctcrypt22}
        % \nocite{dmapp22}
        % \nocite{krasnovidovo23}
        % \nocite{armenia23}
        % \nocite{sibecrypt23}
        % \nocite{sozvezdie24}
        % \nocite{sozvezdie24_2}
        % \nocite{msu24}
    \end{refsection}%
        %
        % Всё, что вне этих двух refsection, это refsection=0,
        %  * для диссертации - это нормальные ссылки, попадающие в обычную библиографию
        %  * для автореферата:
        %     * при usefootcite==0, ссылка корректно сработает только для источника из `external.bib`. Для своих работ --- напечатает "[0]" (и даже Warning не вылезет).
        %     * при usefootcite==1, ссылка сработает нормально. В авторской библиографии будут только процитированные в refsection=0 работы.
}
 % Характеристика работы по структуре во введении и в автореферате не отличается (ГОСТ Р 7.0.11, пункты 5.3.1 и 9.2.1), потому её загружаем из одного и того же внешнего файла, предварительно задав форму выделения некоторым параметрам

%Диссертационная работа была выполнена при поддержке грантов \dots

%\underline{\textbf{Объем и структура работы.}} Диссертация состоит из~введения,
%четырех глав, заключения и~приложения. Полный объем диссертации
%\textbf{ХХХ}~страниц текста с~\textbf{ХХ}~рисунками и~5~таблицами. Список
%литературы содержит \textbf{ХХX}~наименование.

\pdfbookmark{Содержание работы}{description}                          % Закладка pdf
\section*{Содержание работы}

%%%%%%%%%%%%%%%%%%%%
%%%%% ВВЕДЕНИЕ %%%%%
%%%%%%%%%%%%%%%%%%%%

    Во \underline{\textbf{Введении}} обосновывается актуальность исследований, проводимых в~рамках данной диссертационной работы, приводится обзор научной литературы по~изучаемой проблеме, формулируется цель, ставятся задачи работы, излагается научная новизна и практическая значимость представляемой работы. 

%%%%%%%%%%%%%%%%%%%%
%%%%% ГЛАВА 1  %%%%%
%%%%%%%%%%%%%%%%%%%%

    \underline{\textbf{Первая глава}} посвящена введению основных определений, используемых далее на протяжении всей работы.
    Вводится понятие $d$-квазигруппы~--- множества $Q$ с заданной на нем операцией $h \colon Q^d \to Q$ со свойством однозначной разрешимости уравнения $h(x_1, \ldots, x_d) = y$ при фиксации любых $d$ значений переменных $\{x_1, \ldots, x_d, y \}$ относительно свободной переменной.
    Также рассматривается понятие семейства отображений $\ff \colon Q^n \to Q^n$ на $Q^n$, где $\ff = (f_1, \ldots, f_n)$, $f_i \colon Q^n \to Q$,
    \[
        \ff(x_1, \ldots, x_n) = \left(f_1(x_1, \ldots, x_n), \ldots, f_n(x_1, \ldots, x_n) \right),
    \]
    обсуждаются некоторые свойства таких семейств и их возможные преобразования (внешние и внутренние сдвиги, согласованные перестановки, проекции, сужения).
    Семейство $\ff$ на $\EE_2^n$ называется семейством булевых функций.

    Далее вводится основной объект исследования~--- правильное семейство функций: семейство $\ff \colon Q^n \to Q^n$ называется правильным, если для любых неравных наборов $\alpha, \beta \in Q^n$ найдется такой индекс $1 \le i \le n$, что $\alpha_i \ne \beta_i$, но $f_i(\alpha) = f_i(\beta)$.
    Рассматриваются некоторые свойства правильных семейств (булевых) функций.
    В частности, доказана следующая теорема, являющаяся обобщением критерия регулярности~\cite{nosov06abel}: семейство $\ff_n$ на $\QQQ$ является правильным тогда и только тогда, когда для любого набора отображений $\psi_i \colon Q_i \to Q_i$, $1 \le i \le n$, следующее отображение из $\QQQ$ в себя биективно:
    \[
        \xx = 
        \begin{bmatrix}
            x_1\\
            \vdots \\
            x_n \\
        \end{bmatrix} 
        \to
        \xx \circ \Psi(\ff_n(\xx))
        = 
        \begin{bmatrix}
            x_1 \circ_1 \psi_1(f_1(x_1, \ldots, x_n)) \\
            \vdots \\
            x_n \circ_n \psi_n(f_n(x_1, \ldots, x_n))
        \end{bmatrix}, \; x_i \in Q_i.
    \]

    Рассматриваются различные примеры правильных семейств: константные, треугольные, линейные и ортогональные семейства, треугольные расширения.
    Отдельно рассматриваются два семейства специального вида
    \begin{equation}
    \label{example:family1}
        \begin{bmatrix}
            0 \\
            x_1 \\
            x_1 \oplus x_2 \\
            \vdots \\
            x_1 \oplus x_2 \oplus \ldots \oplus x_{n-1}
            \end{bmatrix}
            \bigoplus
            \begin{bmatrix}
            \bigoplus_{i < j, \; i, j \ne 1}^n \; x_i x_j \\
            \bigoplus_{i < j, \; i, j \ne 2}^n \; x_i x_j \\
            \bigoplus_{i < j, \; i, j \ne 3}^n \; x_i x_j \\
            \vdots \\
            \bigoplus_{i < j, \; i, j \ne n}^n \; x_i x_j \\
        \end{bmatrix},
    \end{equation}
    \begin{equation}
        \label{example:family2}
        \ff(\xx) = 
        \begin{bmatrix}
                f_1(x_1, \ldots, x_n) \\
                f_2(x_1, \ldots, x_n) \\
                \vdots \\
                f_n(x_1, \ldots, x_n) \\
            \end{bmatrix}
            =
            \begin{bmatrix}
                \overline{x}_2 \cdot x_3 \\
                \overline{x}_3 \cdot x_4 \\
                \vdots \\
                \overline{x}_1 \cdot x_2 \\
            \end{bmatrix}.
        \end{equation}
    Доказано, что 
    \begin{itemize}
        \item семейства вида~(\ref{example:family1}) являются правильными для любого $n \ge 1$,
        \item семейства вида~(\ref{example:family1}) при $n \ge 3$ является квадратичным строгого типа $Quad^s_{n-1}Lin^s_{1}$ при четных $n$ и квадратичным строгого типа $Quad^s_{n}Lin^s_{0}$ (сильно квадратичным) при нечетных~$n$. 
    \end{itemize}


    Рассматриваются основные свойства квазигрупп, релевантные с точки зрения криптографических приложений: полиномиальная полнота, отсутствие подквазигрупп, количество ассоциативных троек.
    Доказывается ряд утверждений о числе ассоциативных троек в квазигруппах, задаваемых операцией 
    \[
        \xx \circ \yy = \xx + \ff(\xx) + \yy + \gf(\yy), \; \xx, \yy \in \GGG^n, \; (\GGG, +) \text{~--- группа.}
    \]
    В частности, показывается справедливость следующих утверждений:
    \begin{itemize}
        \item тройка $(\xx, \yy,  \mathbf{z})$ является ассоциативной в квазигруппе $(\GGG^n, \circ)$, построенной по паре семейств $(\ff, \gf)$, тогда и только тогда, когда тройка $( \mathbf{z}, \yy, \xx)$ является ассоциативной в квазигруппе, построенной по паре семейств $(\gf,\ff)$;
        \item тройка $(\xx, \yy,  \mathbf{z})$ является ассоциативной для квазигруппы, построенной по паре правильных семейств $(\ff, \gf)$, тогда и только тогда, когда она является ассоциативной для квазигруппы, построенной по паре правильных семейств $(\ff \oplus \alpha, \gf \oplus \alpha)$, где $\alpha \in \ZZ_2^n$;
        \item количество ассоциативных троек в квазигруппе, построенной по паре правильных булевых семейств $(\ff, \gf)$, четно.
    \end{itemize}

    Результаты главы были опубликованы в~\cite{dm21, galatenko23, galatenko2023proper, tsar24}.

    % \begin{theorem}
    %     Пусть $\mathcal{A}$~--- такое обратимое линейное отображение (т.е. $\mathcal{A}(\xx+\yy) = \mathcal{A}(\xx) + \mathcal{A}(\yy)$), что семейства 
    %     \[
    %         \ff'(\xx) = \mathcal{A}^{-1} ( \ff ( \mathcal{A} (\xx) ) ), \quad \gf'(\yy) = \mathcal{A}^{-1} ( \gf ( \mathcal{A} (\yy) ) )
    %     \]
    %     также являются правильными (так, в качестве $\mathcal{A}$ можно рассмотреть преобразование обратимой линейной перекодировки, см. раздел~\ref{sec:reencoding}).
    %     В таком случае $(\xx, \yy,  \mathbf{z})$ является ассоциативной тройкой для квазигруппы, построенной по паре правильных семейств $(\ff, \gf)$, тогда и только тогда, когда тройка $( \mathcal{A}^{-1}(\xx), \mathcal{A}^{-1}(\yy), \mathcal{A}^{-1}( \mathbf{z}) )$ является ассоциативной для квазигруппы, построенной по паре правильных семейств $(\ff', \gf')$.
    % \end{theorem}


% картинку можно добавить так:
% \begin{figure}[ht]
%     \centerfloat{
%         \hfill
%         \subcaptionbox{\LaTeX}{%
%             \includegraphics[scale=0.27]{latex}}
%         \hfill
%         \subcaptionbox{Knuth}{%
%             \includegraphics[width=0.25\linewidth]{knuth1}}
%         \hfill
%     }
%     \caption{Подпись к картинке.}\label{fig:latex}
% \end{figure}

% Формулы в строку без номера добавляются так:
% \[
%     \lambda_{T_s} = K_x\frac{d{x}}{d{T_s}}, \qquad
%     \lambda_{q_s} = K_x\frac{d{x}}{d{q_s}},
% \]


%%%%%%%%%%%%%%%%%%%%
%%%%% ГЛАВА 2  %%%%%
%%%%%%%%%%%%%%%%%%%%

    Во \underline{\textbf{второй главе}} рассматриваются эквивалентные определения правильности. 
    Вводятся понятия 
    \begin{itemize}
        \item одностоковой ориентации булева куба ($\mathsf{USO}$-ориентации),
        \item асинхронной булевой сети с наследственно неподвижной общей точкой ($\hupf$-сети),
        \item графа семейства $\Gamma_{\ff}$.
    \end{itemize}

    Показано, что у правильного семейства булевых функций как отображения $\EE_2^n \to \EE_2^n$ всегда существует единственная неподвижная точка.
    Устанавливается характеристическое свойство булевых правильных семейств: граф семейства $\Gamma_{\ff}$ является одностоковой ориентацией булева куба $\EE_n$ тогда и только тогда, когда $\ff$~--- правильное семейство.

    Полученное соответствие позволяет получить ряд следствий, переведя часть результатов из области $\uso$-ориентаций на язык правильных семейств булевых функций.
    В частности, удается получить ограничение $\log_2 \left( T(n) \right) = \Theta \left( 2^n \cdot log_2(n)\right)$ на число правильных семейств булевых функций $T(n)$.
    Для треугольных семейств удается показать, что их количество $\Delta(n)$ есть о-малое от общего числа всех правильных булевых семейств:
    \[
        \frac{\Delta(n)}{T(n)} = o \left(\frac{1}{n^{D \cdot 2^n}} \right)
        \text{ при } n \to \infty
    \]
    для некоторого $D > 0$.

    Соответствие между $\uso$-ориентациями и правильными семействами позволяет получить новые классы правильных семейств функций.
    Семейство $\ff \colon Q^n \to Q^n$ назовем
    \begin{itemize}
        \item рекурсивно треугольным, если существует такая координата $i$, что $f_i = q \in Q$ (константа), и каждое из семейств вида $\proj^a_i(\ff)$, где $a$ пробегают все множество $Q$, также является рекурсивно треугольным.
        \item локально треугольным, если для каждой точки $\xx \in Q^n$ существует такая согласованная перестановка семейства $\sigma$, что после ее применения мы получим семейство $\gf$ со свойством
        \[
            \dd_i g_j(\xx) = 0, \quad 1 \le j \le i \le n,
        \]
        где $\dd_i (f)$~--- частная производная функции $f$ по направлению $i$.
    \end{itemize}

    Доказан ряд утверждений о введенных классах семейств:
    \begin{itemize}
        \item локально треугольные семейства являются правильными;
        \item класс рекурсивно треугольных семейств вкладывается в класс локально треугольных семейств (в частности, все рекурсивно треугольные семейства являются правильными);
        \item для числа рекурсивно треугольных семейств получена формула
        \[
            \Delta^{\rec}_{k}(n) = \sum_{j=1}^{n} (-1)^{j+1} \cdot k^j \cdot {n \choose j} \left( \Delta^{\rec}_{k}(n-j) \right)^{k^j},
        \]
        где $\Delta^{\rec}_{k}(0) = 1$, $k = \lvert Q \rvert$;
        \item доля булевых рекурсивно треугольных семейств размера $n$ в классе всех булевых правильных семейств размера $n$ стремится к 0 при $n \to \infty$. 
    \end{itemize}

    Также в главе рассматривается кликовое задание правильных семейств.
    Зададим обобщенный граф Келлера $G(k, n)$ следующим образом:
    \begin{itemize}
        \item множество вершин графа $V$~--- наборы чисел от $0$ до $k^2-1$ длины $n$: $V = \EE_{k^2}^n$,
        \item ребро $\{v, w\} \in E$, тогда и только тогда, когда найдется координата $i$, $1 \le i \le n$, что выполнены два условия: $v_i \equiv w_i \mod \; k, v_i \ne w_i$.
    \end{itemize}
    Показано, что правильные семейства $\ff_n$ на $\EE_k^n$ находятся во взаимно-однозначном соответствии с \textbf{кликами} в графе $G(k,n)$ размера $k^n$ (по правильному семейству строится клика в графе $G(k, n)$, и наоборот, по каждой клике размера $k^n$ в $G(k, n)$ задается правильное семейство на $\EE_k^n$).

    Результаты главы ранее были опубликованы в~\cite{pdm20, intsys20, dm21}.



%%%%%%%%%%%%%%%%%%%%
%%%%% ГЛАВА 3  %%%%%
%%%%%%%%%%%%%%%%%%%%



    \underline{\textbf{Третья глава}} посвящена исследованию некоторых свойств правильных (булевых) семейств.
    Решена задача о поиске стабилизатора множества правильных семейств относительно действия 
    \[
        (\Phi, \Psi) \curvearrowright \ff \colon x \to \Phi(\ff(\Psi(x))),
    \]
    где $\Phi, \Psi \in \SSS_{\EE_k^n}$~--- биекции.
    Фактически, показано, что стабилизаторами множества правильных семейств являются <<согласованные>> изометрии пространства Хэмминга.
    
    Доказан ряд результатов для булевых правильных семейств:
    \begin{itemize}
        \item для любого $\alpha \in \EE_2^n$ число решений уравнения $\ff_n(\xx) = \alpha$ всегда четно,
        \item для отображения, задаваемого семейством вида~(\ref{example:family1}), мощность образа максимальна (в классе отображений, задаваемых правильными булевыми семействами) и равна $2^{n-1}$, т.е. семейство имеет максимально возможную мощность образа,
        \item для отображения, задаваемого семейством вида~(\ref{example:family2}), мощность образа равна $\lucas_n$ ($n$-е число Люка).
    \end{itemize}

    Также изучаются некоторые алгебраические свойства отображений, порождаемых правильными семействами функций.
    Будем через $\sprop$ обозначать множество таких подстановок $\sigma \in \SSS_{Q^n}$, что отображение $\ff_n$ вида
    \[
        \ff_n(\xx) = 
        L_{\xx}^{-1} \left( \sigma(\xx) \right) = 
        \begin{bmatrix}
            L_{x_1}^{-1}(\sigma_1(\xx)) \\
            \vdots \\
            L_{x_n}^{-1}(\sigma_n(\xx)) \\
        \end{bmatrix},
    \]
    где $L_{x}(y) = x \circ y$, является правильным на $Q^n$.
    пусть $\ff$~--- правильное семейство на $\GGG^n$, $(\GGG, \cdot)$~--- группа, рассмотрим \textit{дуальное} к семейству $\ff$ семейство $\gf$, соответствующее подстановке $\sigma^{-1}$:
    \[
        \gf \colon Q^n \to Q^n, \quad \gf(\xx) = \xx \cdot \sigma^{-1}(\xx).
    \]

    \begin{itemize}
        \item множество $\sprop$ замкнуто относительно взятия обратной подстановки: отображение $\gf(\xx)$ является правильным на $Q^n$,
        \item замыкание $\langle \sprop \rangle$ действует транзитивно на множестве $Q^n$,
        \item для булева случая показано, что каждая подстановка $\pi \in \sprop$ всегда имеет четное число неподвижных точек.
    \end{itemize}

    Результаты главы ранее были опубликованы в~\cite{pdm20, intsys20, dm21, galatenko23, fpm23}.
    

% Можно сослаться на свои работы в автореферате. Для этого в файле
% \verb!Synopsis/setup.tex! необходимо присвоить положительное значение
% счётчику \verb!\setcounter{usefootcite}{1}!. В таком случае ссылки на
% работы других авторов будут подстрочными.
% Изложенные в третьей главе результаты опубликованы в~\cite{vakbib1, vakbib2}.
% Использование подстрочных ссылок внутри таблиц может вызывать проблемы.


%%%%%%%%%%%%%%%%%%%%
%%%%% ГЛАВА 4  %%%%%
%%%%%%%%%%%%%%%%%%%%

    В \underline{\textbf{четвертой главе}} приведены результаты, касающиеся алгоритмических и вычислительных аспектов.
    Получены следующие результаты:
    \begin{itemize}
        \item предлагается к рассмотрению один алгоритм шифрования, сохраняющего формат (т.н. $\mathsf{FPE}$-алгоритм), основанный на квазигрупповых сдвигах в квазигруппах, порожденных правильными семействами булевых функций,  
        \item 
    \end{itemize}
    Результаты главы ранее были опубликованы в~\cite{fpe22, sibecrypt23, tsar24}.

\FloatBarrier
    \pdfbookmark{Заключение}{conclusion}                                  % Закладка pdf
    В \underline{\textbf{заключении}} приведены основные результаты работы, которые заключаются в следующем:
    %% Согласно ГОСТ Р 7.0.11-2011:
%% 5.3.3 В заключении диссертации излагают итоги выполненного исследования, рекомендации, перспективы дальнейшей разработки темы.
%% 9.2.3 В заключении автореферата диссертации излагают итоги данного исследования, рекомендации и перспективы дальнейшей разработки темы.

    \begin{enumerate}
        \item Установлено естественное соответствие между булевыми правильными семействами и одностоковыми ориентациями графов булевых кубов ($\uso$-ориентации).
        \item Установлено естественное соответствие между булевыми правильными семействами и булевыми сетями с наследственно единственной неподвижной точкой ($\hupf$-сети).
        \item Установлено естественное соответствие между правильными семействами в логике произвольной значности и кликами в обобщенных графах Келлера.
        \item Доказано, что стабилизатором множества правильных семейств функций являются изометрии пространства Хэмминга (согласованные перенумерации и перекодировки).
        \item Показано, что отображения, задаваемые с помощью правильных семейств булевых функций, всегда имеют четное число неподвижных точек.
        \item Получена оценка на число правильных семейств булевых функций, предложены оценки доли треугольных семейств среди всех правильных семейств булевых функций.
        \item Обнаружены и исследованы новые классы правильных семейств функций (рекурсивно треугольные, локально треугольные, сильно квадратичное семейство).
        \item Получены оценки на число рекурсивно треугольных семейств.
        \item Для некоторых правильных семейств булевых функций получены точные значения мощности образа отображений, задаваемых этими правильными семействами.
        \item Предложен новый способ порождения квазигрупп на основе правильных семейств функций.
        \item Доказан ряд утверждений о числе ассоциативных троек в порождаемых квазигруппах.
        \item Предложен новый алгоритм шифрования, сохраняющего формат ($\fpe$-схема), основанный на квазигрупповых операциях.
    \end{enumerate}

    В качестве тем для дальнейших исследований можно отметить следующие направления.

    \begin{enumerate}
        \item Предложить способ построения достаточно широких классов правильных семейств с хорошими алгебраическими и комбинаторными свойствами, в том числе и для логик большей значности $k > 2$.

        \item Предложить способ быстрого построения множества представителей всех правильных семейств размера $n+1$ с помощью представителей размера $n$ и менее (с точностью до согласованных перенумераций и перекодировок).

        \item Предложить альтернативные геометрические описания правильных семейств в $k$-значной логике, где $k>2$, которые были бы инвариантны относительно согласованных перенумераций и перекодировок.

        \item Предложить алгоритм, полиномиальный по длине входа, на вход принимающий правильное семейство (например, в виде КНФ или полиномов Жегалкина) и параметрические подстановки и выдающий количество ассоциативных троек (или нижние и верхние границы на число троек), проверяющий полиномиальную полноту порождаемой квазигруппы, наличие или отсутствие подквазигрупп.

        \item Оценить генерическую сложность задачи решения системы уравнений над квазигруппами, заданными правильными семействами.
    \end{enumerate}


\pdfbookmark{Литература}{bibliography}                                % Закладка pdf
% При использовании пакета \verb!biblatex! список публикаций автора по теме
% диссертации формируется в разделе <<\publications>>\ файла
% \verb!common/characteristic.tex!  при помощи команды \verb!\nocite!

\ifdefmacro{\microtypesetup}{\microtypesetup{protrusion=false}}{} % не рекомендуется применять пакет микротипографики к автоматически генерируемому списку литературы
\urlstyle{rm}                               % ссылки URL обычным шрифтом
\ifnumequal{\value{bibliosel}}{0}{% Встроенная реализация с загрузкой файла через движок bibtex8
    \renewcommand{\bibname}{\large \bibtitleauthor}
    \nocite{*}
    \insertbiblioauthor           % Подключаем Bib-базы
    %\insertbiblioexternal   % !!! bibtex не умеет работать с несколькими библиографиями !!!
}{% Реализация пакетом biblatex через движок biber
    % Цитирования.
    %  * Порядок перечисления определяет порядок в библиографии (только внутри подраздела, если `\insertbiblioauthorgrouped`).
    %  * Если не соблюдать порядок "как для \printbibliography", нумерация в `\insertbiblioauthor` будет кривой.
    %  * Если цитировать каждый источник отдельной командой --- найти некоторые ошибки будет проще.
    %
    % %% authorvak
    % \nocite{vakbib1}%
    % \nocite{vakbib2}%
    % %
    % %% authorwos
    % \nocite{wosbib1}%
    % %
    % %% authorscopus
    % \nocite{scbib1}%
    % %
    % %% authorpathent
    % \nocite{patbib1}%
    % %
    % %% authorprogram
    % \nocite{progbib1}%
    % %
    % %% authorconf
    % \nocite{confbib1}%
    % \nocite{confbib2}%
    % %
    % %% authorother
    % \nocite{bib1}%
    % \nocite{bib2}%
    \nocite{intsys20}
    \nocite{pdm20}
    \nocite{dm21}
    \nocite{fpe22}
    \nocite{galatenko23}
    \nocite{galatenko2023proper}
    \nocite{fpm23}
    \nocite{tsar24}

    \ifnumgreater{\value{usefootcite}}{0}{
        \begin{refcontext}[labelprefix={}]
            \ifnum \value{bibgrouped}>0
                \insertbiblioauthorgrouped    % Вывод всех работ автора, сгруппированных по источникам
            \else
                \insertbiblioauthor      % Вывод всех работ автора
            \fi
        \end{refcontext}
    }{
        \ifnum \totvalue{citeexternal}>0
            \begin{refcontext}[labelprefix=A]
                \ifnum \value{bibgrouped}>0
                    \insertbiblioauthorgrouped    % Вывод всех работ автора, сгруппированных по источникам
                \else
                    \insertbiblioauthor      % Вывод всех работ автора
                \fi
            \end{refcontext}
        \else
            \ifnum \value{bibgrouped}>0
                \insertbiblioauthorgrouped    % Вывод всех работ автора, сгруппированных по источникам
            \else
                \insertbiblioauthor      % Вывод всех работ автора
            \fi
        \fi
        %  \insertbiblioauthorimportant  % Вывод наиболее значимых работ автора (определяется в файле characteristic во второй section)
        \begin{refcontext}[labelprefix={}]
            \insertbiblioexternal            % Вывод списка литературы, на которую ссылались в тексте автореферата
        \end{refcontext}
        % Невидимый библиографический список для подсчёта количества внешних публикаций
        % Используется, чтобы убрать приставку "А" у работ автора, если в автореферате нет
        % цитирований внешних источников.
        \printbibliography[heading=nobibheading, section=0, env=countexternal, keyword=biblioexternal, resetnumbers=true]%
    }
}
\ifdefmacro{\microtypesetup}{\microtypesetup{protrusion=true}}{}
\urlstyle{tt}                               % возвращаем установки шрифта ссылок URL
